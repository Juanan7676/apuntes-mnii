\documentclass[11pt]{article}
\usepackage[spanish]{babel}
\usepackage[utf8]{inputenc}

% ------ Cargar estilo específico para la relación de problemas
\usepackage{problemas-MNII}

% ==============
\begin{document}
% =============
\begin{flushright}
  \LARGE\it Relación de problemas. Tema \huge 1.\\
  \bigskip
\end{flushright}

\begin{problemas}

  \begin{problema}
    Consideremos el polinomio
    $$
    f(x) = x^3 + 4x^2 -3x -5.
    $$
    Demostrar que $f(x)$ tiene una única raíz positiva y aproximarla mediante el
    método de bisección, de forma que el error absoluto sea menor que $10^{-2}$.
  \end{problema}
  \begin{problema}
    Nos proponemos encontrar la solución de la siguiente ecuación en
    el intervalo $[0,\pi]$:
    $$
    \int_0^x (\sen s)^2 \, ds = 1.
    $$

    \begin{enumerate}
    \item Demostrar que el problema es equivalente a hallar un punto
      fijo de la función:
      $$
      g(x)=2 + \frac{\sen(2x)}{2}
      $$
      en el intervalo $[0,\pi]$.
    \item Comprobar que la función $g$ satisface las hipótesis del
      teorema del punto fijo en el intervalo $\displaystyle [1.6,2]\subset
      \bigg(\frac{\pi}{2},\frac{3\pi}{4}\bigg)$.
    \item Determinar el número de iteraciones, $k$, para las que el
      error cometido sea menor que $10^{-8}$.
      \begin{quote}\em\small
        Indicación:
        $|x_k-\alpha|\le \lambda^k|x_0-\alpha|,
        \quad\lambda=\text{c. de contractividad}$
      \end{quote}
    \item Calcular las dos primeras iteraciones.
    \end{enumerate}
  \end{problema}

  % \begin{problema}
  %   Dada la ecuación $f(x)=e^x-(x-2)^2=0$, demostrar que sólo posee una
  %   raíz real.
  %   \begin{flushright}\em\small
  %     Idea: probar que $f'(x)>0$ para todo $x\in\Rset$. Para ello,
  %     cuando $x>2$, aplicar a $e^x$ el teorema del valor medio en
  %     $[2,x]$ para probar que $e^x>2(x-2)$.
  %   \end{flushright}
  % \end{problema}

  % \begin{problema}
  %   Consideremos la ecuación $$f(x)=(x-1)\tan(x)-1=0.$$
  %   \begin{enumerate}
  %   \item Demostrar esta ecuación posee infinitas raíces positivas,
  %     determinando los intervalos que contengan a cada una de ellas.
  %   \item Sea $\alpha>0$ la menor raíz positiva de $f(x)$ y
  %     consideremos el intervalo $[a,b]=[0,\pi/4]$. Reescribir la
  %     ecuación anterior en términos de punto fijo, $x=g(x)$, para
  %     $g(x)$ adecuada. Demostrar que $\alpha\in [a,b]$ y que $g(x)$
  %     verifica las hipótesis del teorema del punto fijo.
  %   \item Partiendo de $x_0=0$, realizar dos iteraciones del método
  %     de punto fijo. Determinar $n\in\Nset$ tal que, a partir de
  %     $x_n$, el error absoluto con la solución exacta sea menor que
  %     $\varepsilon = 10^{-3}$.
  %   \end{enumerate}
  % \end{problema}

  \begin{problema}
    Se sesea aplicar el método de punto fijo para calcular las raíces
    de la ecuación
    $$
    f(x)=2x^2+6e^{-x}-4=0.
    $$
    \begin{itemize}
    \item Reescribir (para cada raíz) la ecuación anterior en términos
      de punto fijo y demostrando que se verifican las hipótesis del
      teorema correspondiente.
    \item Realizar (para cada raíz) cinco iteraciones de punto fijo.
    \item Estimar el error cometido, en cada caso.
    \end{itemize}
  \end{problema}

  \begin{problema}
    Fijado $\gamma>0$, podemos usar el método de Newton para calcular
    raíces reales $n$--ésimas (para cualquier $n\in\Nset$ dado)
    de $\gamma$, resolviendo la ecuación:
    $$
    x^n - \gamma = 0.
    $$
    \begin{enumerate}
    \item Determinar un intervalo y un valor inicial para los que el
      método de Newton sea convergente
      \begin{quote}\em\small
        Indicación: distinguir los casos $\gamma>1$ y $\gamma<1$.
      \end{quote}
    \item Realizando tres iteraciones del método de Newton, aproximar
      los valores de $\sqrt{2}$ y $\sqrt[3]{1/3}$. Estimar, en cada
      caso, el error cometido.
    \end{enumerate}

  \end{problema}


  \begin{problema}
    Consideremos la ecuación
    $$
    x-1 + \log(1+x^2) = 0
    $$
    \begin{enumerate}
    \item Demostrar que existe una única solución, $\alpha\in\Rset$.
    \item Comprobar que se verifican las hipótesis de convergencia
      global del método de Newton en el intervalo $[0,1]$, pero no
      las hipótesis de la regla de Fourier (que es una condición
      suficiente pero no necesaria).
    \item Realizar tres iteraciones del método de Newton a partir de
      $x_0=1$. Si $x_3$ es la aproximación obtenida, acotar el error
      $|x_3-\alpha|$ en función de $|x_1-x_0|$.
    \end{enumerate}
  \end{problema}

  \begin{problema}
    Dada la ecuación $$e^x=(x+1)^2:$$
    \begin{enumerate}
    \item Estudiar sus raíces y localizarlas en intervalos
    \item Utilizar el método de bisección para aproximar la mayor de
      ellas con dos cifras decimales exactas
    \item Para esta raíz, localizar un intervalo en el que, mediante
      la regla de Fourier, se pueda garantizar la convergencia del
      método de Newton. Utilizar este método para aproximar esta raíz
      con cuatro cifras decimales exactas.
    \end{enumerate}
  \end{problema}
\end{problemas}


\end{document}

%%% Local Variables:
%%% mode: latex
%%% TeX-master: t
%%% End:
