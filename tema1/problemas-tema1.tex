\documentclass[11pt]{article}
\usepackage[spanish]{babel}
\usepackage[utf8]{inputenc}

% ------ Cargar estilo específico para la relación de problemas
\usepackage{problemas-MNII}

% ==============
\begin{document}
% =============

\begin{problemas}
  \begin{problema}
    Demostrar que la función $f(x)=x^2e^{x}-1$ tiene un único
    cero. Utilizar el método de bisección para aproximarlo con un
    error menor que $\varepsilon=10^{-7}$. Comparar el resultado con
    la solución aproximada que proporciona la función \texttt{fsolve}
    (contenida en el paquete \texttt{optimize} de \textit{Scipy}).
  \end{problema}
  \begin{problema}
    Utilizar el método de bisección para aproximar las raíces de las
    siguientes ecuaciones con una tolerancia menor que
    $\varepsilon=10^{-8}$.
    \begin{enumerate}
    \item $e^x-3x^2=0$.
    \item $x^3=x^2+x+1$.
    \end{enumerate}
  \end{problema}
 \begin{problema}
    Consideremos el siguiente producto infinito para el cálculo de $\pi$:
    (publicado en 1593 por François Viète y conocido como fórmula de Viète):
    \begin{equation*}
      \frac{2}{\pi}  = \prod_{k\to\infty} \frac{a_k}2, \qquad\text{donde }\quad
      a_0 = \sqrt{2}, \quad a_{k+1} = \sqrt{2+a_k}.
    \end{equation*}

    \begin{enumerate}
    \item Calcular las $15$ primeras aproximaciones de $\pi$ para la
      fórmula de Viète (correspondientes a la evaluación del producto
      anterior para $k=0,...,14$) y sus errores absolutos, $e_k$.
    \item Mostrar en una tabla los logaritmos de los errores, $\log
      e_k$, y los cocientes $\log e_{k+1}/\log
      e_{k}$ (con $k\ge 0$).
    \item Justificar la siguiente afirmación y utilizarla para estimar
      el orden de la fórmula de Viète:
      \begin{equation*}
        \label{eq:1}
        \{x_k\} \text{ es un método iterativo de orden } p 
        \Leftrightarrow \lim_{k\to\infty} \frac{\log e_{k+1}}{\log e_{k}}=p.
      \end{equation*}
    \end{enumerate}
  \end{problema}
  \begin{problema}
    Consideremos el problema de punto fijo $x=g(x)$, con
    $g(x)=1/(2+x)$.  Probar que existe una única solución en el
    intervalo $[0,1]$ y que método de aproximaciones sucesivas
    $x_{k+1}=g(x_k)$ converge hacia una solución para cualquier
    inicialización, $x_0\in[0,1]$. Aproximar la
    solución con un error menor que $\varepsilon=10^{-8}$.
  \end{problema}
  \begin{problema}
    Dada la ecuación $f(x)=e^x-(x+1)^2=0$:
    \begin{itemize}
    \item Estudiar sus raíces reales
    \item Localizarlas en intervalos en los que el método de Newton
      esté bien definido y sea convergente (idea: en cada intervalo,
      aplicar la regla de Fourier)
    \item En cada uno de estos intervalos, calcular una aproximación
      de la raíz con un error menor a $10^{-6}$.
    \end{itemize}
  \end{problema}
  \begin{problema}
    Dada la ecuación $f(x)=e^x-(x-2)^2=0$:
    \begin{itemize}
    \item Justificar geométricamente que sólo posee una raíz real
    \item Demostrarlo analíticamente. Idea: probar que $f'(x)>0$ para
      todo $x\in\Rset$. Para ello, cuando $x>2$, aplicar a $e^x$ el
      teorema del valor medio en $[2,x]$ para probar que $e^x>2(x-2)$.
    \end{itemize}
  \end{problema}
  \begin{problema}
    Calcular las tres primeras aproximaciones ($k=0,1,2$) de $\pi$
    mediante el algoritmo de Gauss-Legendre (o de Brent-Salamin):
    $a_0=1$, $b_0=1/\sqrt 2$, $t_0=1/4$, $p_0=1$,
    \begin{align*}
      a_{k+1}&=(a_{k}+b_{k})/2 , &\quad
      t_{k+1}&=t_k - p_k(a_k -a_{k+1})^2, \\
      b_{k+1}& =\sqrt{a_{k}\*b_{k}}, &\quad p_{k+1}&=2p_k.
    \end{align*}
    $$
    \pi = \lim_{k\to\infty} \frac{(a_k+b_k)^2}{4t_k}.
    $$
    Utilizando los logaritmos de los errores absolutos, estimar el
    orden del algoritmo.
  \end{problema}
  \begin{problema}
    Consideremos el método de bisección para
    una función $f$ en el intervalo $[a,b]=[2.6,4.2]$. 
    \begin{enumerate}
    \item ¿Cuál es la longitud del subintervalo en el paso $k$--ésimo?
    \item ¿Cuál es la máxima distancia entre la raíz, $\alpha$, de $f$
      y el punto medio del intervalo en ese paso $k$--ésimo?
    \item ¿Cuántas iteraciones son necesarias, como mínimo, para que
      la longitud del intervalo sea menor que $\varepsilon=10^{-5}$?
    \end{enumerate}
  \end{problema}
 
\end{problemas}
\end{document}

%===============

%%% Local Variables: 
%%% mode: latex
%%% TeX-master: t
%%% End: 
