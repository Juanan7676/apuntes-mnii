\documentclass[11pt]{article}
\usepackage[spanish]{babel}
\usepackage[utf8]{inputenc}

% ------- Formato del documento ----
\usepackage[hmargin={2.50cm,2.5cm}, vmargin={3cm,2.5cm}]{geometry}

% ------ Cargar estilo de los apuntes 
\usepackage{apuntes-MNII}


% ==============
\begin{document}
% ==============

\begin{enumerate}
\item Consideremos la sucesión:
  \begin{align*}
    a_{n+1}&=(a_{n}+b_{n})/2 , &\quad
    t_{n+1}&=t_n - p_n(x_n -x_{n+1})^2, \\
    b_{n+1} &=\sqrt{a_{n}\*b_{n}}, 
    &\quad p_{n+1}&=2p_n,
  \end{align*}
con $a_0=1$, $b_0=1/\sqrt 2$, $t_0=1/4$, $p_0=1$.
\item Demostrar que en un método iterativo de orden $p$, el número de
  cifras exactas obtenidas en cada iteración (con respecto a la
  solución exacta, $\alpha$), se multiplica por $p$ en cada iteración.
  \begin{flushright}
    \scriptsize \textit{Indicación}: demostrar que si
    $d_k=-\log_{10}|x_k-\alpha|$, entonces $d_{n+1}=p d_n + r$, con
    $r=-\log_{10}\lambda$, siendo $\lambda$ la constante asintótica de error.
  \end{flushright}
  Para un método de orden uno y $\lambda=0.999$, ¿cuántas iteraciones
  serán necesarias para obtener una nueva cifra exacta? Para orden
  $p=1.01$, ¿cada cuántas iteraciones se dobla el número de cifras
  exactas? ¿Y para $p=1.1$? Demostrar que la sucesión $x_k=2^{-k}$
  proporciona una aproximación de orden $1$ de $\alpha=0$ y determinar
  $\lambda$. Comprobar numéricamente que $d_{n+1}=p\, d_n + r$, para
  distintos valores de $n$.
\end{enumerate}
\end{document}

%===============

%%% Local Variables: 
%%% mode: latex
%%% TeX-master: t
%%% End: 
