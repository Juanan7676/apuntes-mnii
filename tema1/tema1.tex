
\section{Introducción. Orden de convergencia}
\label{sec:intro-orden-convergencia}

En matemáticas y en disciplinas relacionadas con el cálculo científico
aparece con frecuencia el problema de hallar los \textit{ceros de una
  función}.
\begin{center}
  \begin{tikzpicture}
    \begin{axis}[ \axisXYmiddle, xtick=\empty, ytick=\empty, legend
      pos = south east ]
      % Draw a curve
      \addplot[domain=-1:3, blue, ultra thick] {-(x+3)*(x-1)*(x-5)};
      % Plot a label at curve root
      \node[coordinate, medium dot, pin=-30:{$\cero$}] at (axis
      cs:1,0) {};
      % Draw the name of curve
      \legend {$f$};
    \end{axis}
  \end{tikzpicture}
\end{center}

Recordemos que, dada una función real de una variable real, $f$, con
dominio $I\subset\Rset$, una \textit{raíz} (o un \textit{cero}) de $f$
es una solución del siguiente problema:
\begin{equation}
  \label{eq:raiz}
  \tag{P}
  \text{Hallar $\cero\in I$ tal que} \quad f(\cero)=0.
\end{equation}

% El cero es simple si
% $f'(\cero)=0$ y múltiple si $f'(\cero)=0$.

Este no es un problema sencillo y, salvo en algunos casos concretos no
disponemos de algoritmos que nos permitan obtener raíces de una
ecuación en un número finito de pasos. Por ejemplo, no existen
fórmulas explícitas para hallar ceros de funciones polinómicas
arbitrarias de grado $n\ge 5$ y el problema es aún más difícil si $f$
no es polinómica. En general, no podemos plantearnos el hallar las
raíces exactas de una ecuación. Más aún cuando en numerosos <<problemas
reales>>, esta función es conocida sólo de forma aproximada.

Recurriremos a métodos numéricos que, usualmente, vendrán dados en
forma de \textit{algoritmos iterativos}, es decir, partiendo de uno o
más datos iniciales, intentaremos construir una sucesión
$\{x_k\}_{k=0}^{\infty}$ tal que
$$
x_k \to \cero,
$$
done $\cero$ es una raíz de $f$. La idea es elegir $N\in\Nset$
<<suficientemente grande>> de forma que $x_N$ sea <<una buena
aproximación>> de $\cero$; escribiremos
$$
\cero\approx x_N.
$$

El \textit{error de truncamiento} (o simplemente, el
\textit{error}) del método en la etapa $k$ se define como:
$$
e_k = |x_k - \alpha|.
$$

Nuestro objetivo es estudiar algoritmos eficientes que nos permitan
determinar de forma aproximada los ceros de una función $f$ en un
intervalo $I=[a,b]$, de tal forma que el error de truncamiento
respecto a la solución exacta sea tan pequeño como deseemos.

Como veremos, no existe <<\textit{el mejor método}>> para el cálculo
de ceros de funciones: algunos son más rápidos, otros requieren una
buena estimación inicial de la raíz, o regularidad en la función... En
general, un buen método será \emph{muy general} (es decir, podrá
utilizarse para un rango de funciones muy amplio) y a la vez
\emph{poco costoso} (exigirá pocos recursos de ordenador hacer pequeño
el error con la solución exacta). En general identificaremos los
recursos computacionales con el número de operaciones requeridas por
el algoritmo\footnote{También con la cantidad de memoria en el
  ordenador utilizada, aunque en los algoritmos para resolución de
  ecuaciones 1D suelen poco exigente en cuanto a requerimientos de
  memoria.}. Pero en general, la naturaleza de la función $f$ se
escapa de nuestro control y no conoceremos cuantas operaciones
requiere su evaluación.  Por tanto, será frecuente el \emph{estimar el
  coste del algoritmo en términos del número de evaluaciones de $f$}
(y no del número de operaciones en coma flotante).

% En todo caso, el precio de un aumento de los recursos empleados por un
% algoritmo nos puede compensar si, a cambio, aumenta el orden de
% convergencia.


\subsection*{Orden de convergencia}

Consideremos un método iterativo definido por una sucesión $\{x_k\}$
tal que $\lim x_k=\alpha$.
Para medir la ``rapidez con la que converge el método''
utilizaremos el concepto de orden de convergencia.

\begin{definition}
  \label{def:orden-convergencia}
  Sea $\{x_k\}$ una sucesión convergente a $\alpha$ y sea $p\ge
  1$.
  Decimos que la sucesión tiene \resaltar{orden de convergencia
    exactamente igual} a $p\ge 1$ si existe una constante $C>0$ (y
  $C<1$ si $p=1$) tal que
  \begin{equation}
    \label{eq:orden-convergencia}
    \lim_{k\to+\infty} \frac{e_{k+1}}{e_k^p} = C.
  \end{equation}
  En este caso, a $C$ se le llama constante asintótica de error.
\end{definition}

\begin{remark}
  Asumiendo que $x_k \to \alpha$, tendremos que $e_k \to 0$ (y en
  particular existe $k_o$ tal que $0<e_k<1$, para todo $k\ge
  k_o$).
  Cuando La ecuación~(\ref{eq:orden-convergencia}) se puede
  interpretar en el siguiente sentido: en cada iteración, el error
  disminuye como la potencia $p$ del error en la iteración anterior.
  Este hecho lo podemos escribir, de forma rigurosa, como:
  \begin{equation}
    \label{eq:orden-convergencia-aprox}
    e_{k+1} \approx C e_k^p
  \end{equation}
  (en el caso en el que $p=1$, imponemos que $C<1$ para que la
  sucesión de errores tienda a cero).

  Para dar un sentido más preciso a la
  expresión~(\ref{eq:orden-convergencia-aprox}), obsérvese que una
  sucesión con orden de convergencia $p$
  verifica~\eqref{eq:orden-convergencia}, es decir que para todo
  $\varepsilon>0$ existe $k_\epsilon\in\Nset$ tal que si
  $k\ge k_\epsilon$ se tiene
  ${e_{k+1}}/{e_k^p} \in (C-\varepsilon,C+\varepsilon)$.  Por tanto, el
  error en la etapa $k+1$ está acotado superior e inferiormente por el
  error en la etapa $k$:
    \begin{align*}
      (C-\varepsilon) e_k^p < e_{k+1} <
      (C+\varepsilon)  e_k^p, \quad \forall k \ge k_\epsilon.
    \end{align*}
    Nótese que $(C-\varepsilon)$ y $(C+\varepsilon)$ convergen a $C$
    cuando $\varepsilon\to 0$.
    % La desigualdad anterior significa que (fijando la notación
    % $C^\flat_\varepsilon=C-\varepsilon$ y
    % $C^\sharp_\varepsilon=C+\varepsilon$), el error en la etapa $k+1$
    % está acotado inferior y superiormente por el error en la etapa $k$
    % de la siguiente forma:
    % \begin{align*}
    %   C^\flat_\varepsilon \; e_k^p \le e_{k+1},\\
    %   e_{k+1} \le C^\sharp_\varepsilon \; e_k^p.
    % \end{align*}
\end{remark}

En el caso $p=1$ decimos que la convergencia es (exactamente)
\textit{lineal}. Los casos $p>1$ se llaman de convergencia
\textit{superlineal} (cuadrática para $p=2$, supercuadrática si $p>2$,
cúbica para $p=3$, etc.).

\begin{example}
  \label{rk:2}
  Es muy sencillo comprobar que la sucesión $x_k=2^{-k}$ converge a cero
  con orden de convergencia lineal ($p=1$), con constante asintótica $C=1/2$.
\end{example}


% el que exige $C<1$ para garantizar la disminución del error)
Pero en general, el determinar de forma exacta el orden de
convergencia de un algoritmo no es una tarea trivial. Por suerte,
habitualmente, no será necesario hacerlo y bastará con calcular el
orden en un sentido más débil.

En concreto, es fácil comprobar que si una sucesión tiene orden de
convergencia exactamente igual a $p$, entonces existen una constante a
la que también llamaremos $C>0$ y $k_0\in\Nset$ tales que el error en
la etapa $k+1$ está acotado por el error en la etapa anterior de la
siguiente forma:
\begin{equation}
  \label{eq:orden-convergencia-al-menos-p}
  e_{k+1} \le C e_k^p, \quad \forall k\ge k_0.
\end{equation}

\begin{definition}
  Diremos que un método iterativo (definido por una sucesión $\{x_k\}$
  tal que $\lim x_k=\alpha$) tiene \resaltar{orden de convergencia}
  (al menos) $p\ge 1$ si existen una constante $C>0$ y un número
  entero $k_0>0$ para los que se
  verifica~(\ref{eq:orden-convergencia-al-menos-p})
  \label{def:orden-convergencia-al-menos-p}
\end{definition}


\begin{remark}
  Algunas propiedades:
  \begin{enumerate}
  \item Como se ha comentado, toda sucesión de orden exactamente igual
    a $p$ (es decir, que verifique~(\ref{eq:orden-convergencia}))
    también verifica~(\ref{eq:orden-convergencia-al-menos-p}), es
    decir, tiene orden $p$.
  \item De hecho, se verifica la siguiente propiedad: si un método
    iterativo es de orden exactamente $p$, entonces es de orden $q$,
    para todo $1\le q \le p$. La demostración de esta propiedad se
    deja como ejercicio.
  \end{enumerate}
\end{remark}



% \begin{definition}
%   Diremos que un método iterativo (definido por una sucesión $\{x_k\}$ tal
%   que $\lim x_k=\alpha$) tiene \resaltar{orden de convergencia (al menos)}
%   $p\ge 1$ si existen una constante $C>0$ y un número entero $k_0>0$
%   tal que
%   \begin{equation}
%     \label{eq:orden-convergencia-al-menos-p}
%     e_{k+1} \le C e_k^p \quad \forall k\ge k_0.
%   \end{equation}
%   \label{def:orden-convergencia-al-menos-p}
% \end{definition}

% % \begin{remark}
% %   La desigualdad~(\ref{eq:orden-convergencia-2}) significa que todo
% %   método iterativo con orden de convergencia (exactamente) $p$ tiene
% %   orden de convergencia al menos $p$ para una constante ligeramente
% %   mayor, $C+\epsilon$.
% Algunos comentarios:
% \begin{itemize}
% \item Un método iterativo de orden (al menos) $p$ podría tener,
%   exactamente, un orden mayor que $p$. Visto de otra forma, un método
%   iterativo de orden exactamente $p$ es de orden (al menos) $q$, para
%   todo $1\le q < p$ ($C<1$ si $q=1$).
%   \begin{flushright}
%     \vspace{-0.75em}
%     \scriptsize \em La demostración de lo anterior se deja como ejercicio.
%     \vspace{-0.75em}
%   \end{flushright}
% \item En relación con lo anterior, aunque la sucesión $x_k$
%   verifique la desigualdad~(\ref{eq:orden-convergencia-al-menos-p}),
%   no tiene por qué existir el límite~(\ref{eq:orden-convergencia}). En
%   concreto de~(\ref{eq:orden-convergencia-al-menos-p}) sólo podemos
%   concluir que
%   \begin{equation*}
%     \limsup_{k\to+\infty} \frac{e_{k+1}}{e_k^p} < +\infty.
%   \end{equation*}
%   De hecho, no es difícil comprobar que la condición anterior es no
%   solo necesaria, sino suficiente para que la sucesión $x_{k}$ tenga
%   orden (al menos) $p$.
% \item Análogamente a la
%   observación~\ref{rk:interpretacion-orden-convergencia}, la
%   desigualdad~(\ref{eq:orden-convergencia-al-menos-p}) garantiza que
%   en un esquema de orden (al menos) $p$ tiene lugar el decrecimiento del
%   error absoluto en términos de su potencia $p$ (o del factor
%   $C<1$ si $p=1$).
% \end{itemize}

El resto de este capítulo se estructura de la siguiente forma: En una
primera sección, repasamos algunos resultados teóricos para asegurar
la existencia y unicidad de solución de~(\ref{eq:raiz}). En el resto
de las secciones se estudian distintos métodos iterativos para el
cálculo de ceros de funciones de una variable, partiendo del que es,
conceptualmente, más sencillo (el método de Bisección) y llegando
hasta el método de Newton y algunas de sus variantes.  La última
sección ofrece algunas indicaciones sobre la generalización de estos
métodos para resolver sistemas de ecuaciones no lineales.

\section{Existencia y unicidad de solución. Separación de ceros}
\label{sec:tema1:exist-y-unic}

Antes abordar la resolución numérica de cualquier problema es
fundamental el realizar, en una etapa previa, un análisis del mismo
que nos permita determinar la existencia de solución. En caso
afirmativo, en la mayor parte de los casos será muy importante
asegurarnos de que esta solución es única\footnote{Por ejemplo,
  la falta de unicidad de sistemas de ecuaciones lineales está
  asociada con la singularidad de las matrices asociadas. O, en
  métodos iterativos, la no unicidad puede dar pie a
  oscilaciones espurias, es decir, a sucesiones oscilantes que no
  converjan a una solución.}.

Con frecuencia, el proceso previo para determinar los ceros de una
función consiste en localizar intervalos en los que podamos
garantizar la existencia de una única solución (proceso que recibe el
nombre de \textit{separación de ceros o de soluciones}). Los
siguientes resultados nos proporcionan unas herramientas muy útiles
para ello.

Como primera aproximación, para localizar los ceros puede resultar
útil la representación gráfica de la función, aunque \textit{en
  ningún caso debemos confiar exclusivamente en las gráficas generadas
  por programas informáticos}: éstas deben estar respaldadas por un
análisis que garantice la existencia y unicidad de solución.

En general, este análisis se apoya en siguientes teoremas:
\begin{theorem}[Bolzano]
  \label{thm:bolzano}
  Sea $f:[a,b]\subset \Rset\to\Rset$ una función continua en el
  intervalo $[a, b]$ y supongamos que $f (a)\cdot f (b) < 0$.
  Entonces, existe $c\in(a, b)$ tal que $f (c) = 0$.
\end{theorem}

\begin{theorem}[Rolle]
  \label{thm:rolle}
  Sea $f:[a,b]\subset \Rset\to\Rset$ una función continua en $[a, b]$ y derivable en
  $(a, b)$ tal que $f(a) = f(b)$.
  Entonces existe al menos un valor $c \in (a, b)$ tal que $f'(c) = 0$.
\end{theorem}

El teorema de Bolzano proporciona existencia de solución,
pero no unicidad (una función continua con distinto signo en
$a$ y en $b$ podría tener muchos ceros en $(a,b)$). El teorema de
Rolle puede ser utilizado para obtener el siguiente resultado de
unicidad:

% \end{remark}

% Combinando ambos teoremas, podemos deducir el siguiente resultado de
% existencia y unicidad:

\begin{corollary}
  \label{cor:tema1:exist+unic}
  Sea $f:[a,b]\subset \Rset\to\Rset$ continua en $[a, b]$ y derivable
  en $(a, b)$. Si $f'(x)\ne 0$, para todo $x\in (a, b)$, entonces
  existe a lo sumo una solución de~\eqref{eq:raiz} en el intervalo
  $I=[a,b]$.
\end{corollary}

\begin{proof}
  Por reducción al absurdo, si existieran dos soluciones
  de~\eqref{eq:raiz}, $\cero_1$ y $\cero_2$ (con $\cero_1<\cero_2$),
  tendríamos que $0=f(\cero_1)=f(\cero_2)$. Puesto que $f$ se
  encuentra en las hipótesis del teorema de Rolle en el subintervalo
  $[\cero_1,\cero_2]$, esto implicaría que existe $c$ entre $\cero_1$
  y $\cero_2$ en el que $f'(c)=0$, lo que contradice las hipótesis.
\end{proof}

\begin{remark}
  \label{rk:tema1:exist+unic}

  El teorema anterior puede generalizarse a intervalos de la forma
  $(-\infty,b]$, $[a,+\infty)$ o $(-\infty,+\infty)$. En efecto, en
  estos casos podríamos repetir la demostración anterior (si
  existieran dos raíces, $\cero_1<\cero_2$, el teorema de Rolle
  implicaría que existe $c\in (\cero_1,\cero_2)$ tal que $f'(c)=0$).
\end{remark}

En realidad, el corolario anterior se basa en el hecho de que (como
consecuencia del Teorema de Rolle), entre dos raíces de $f$ hay al
menos una raíz de $f'$.

\begin{remark}[Separación de ceros]
  \label{rk:tema1:separac-ceros}
  % algoritmo para la separación de
  % soluciones de~\eqref{eq:raiz} que será utilizado en los próximos
  % ejemplos.
  Suponiendo continuidad y derivabilidad de $f$ (para
  aplicar el Corolario~\ref{cor:tema1:exist+unic}), % suponemos que la
  % derivada, $f'$, \emph{es continua en $(a,b)$},
  podemos enunciar un algoritmo para la localización de las raíces de
  $f$:
  \begin{enumerate}
  \item Hallar todas las raíces de $f'$, a las que llamaremos
    $\beta_1<\dots<\beta_n$.
  \item Considerar los intervalos $(a,\beta_1)$, $(\beta_1,\beta_2)$,
    $(\beta_2,\beta_3)$, \dots,$(\beta_n,b)$. Puesto que, dentro de
    ellos, $f'(x)\neq 0$, sabemos (según el Corolario~\ref{cor:tema1:exist+unic})
    que existe, a lo sumo, una raíz de $f$ en cada uno de estos
    intervalos.
  \item Utilizar el teorema de Bolzano para detectar en cuales de
    estos intervalos existe realmente una raíz de $f$.
  \end{enumerate}
  El proceso anterior es válido incluso si $a=-\infty$ o $b=+\infty$,
  según la observación~\ref{rk:tema1:exist+unic}.  Sin embargo, aunque es
  útil desde el punto de vista teórico y en algunos casos
  particulares, presenta un importante inconveniente en la práctica:
  la necesidad de determinar las raíces de  $f'$.  En
  realidad, estamos convirtiendo el problema del cálculo de ceros de
  $f$ en otro similar, el cálculo de ceros de $f'$, que podría ser tan
  complicado como el anterior (o más aún).

  Con frecuencia, se utilizan otros razonamientos para determinar
  todos los intervalos, $(\beta_i,\beta_{i+1})$ en los que
  $f(\beta_i)\cdot f(\beta_{i+1})<0$. Por ejemplo, procedimientos de
  tipo bisección (que analizaremos en la próxima sección).
\end{remark}

A continuación, veremos algunos ejemplos en los que se aplican los
resultados anteriores:

\begin{example}
  \label{ex:tema1:separ-soluc-1}
  Nos planteamos el hallar las raíces de la ecuación
  $$
  f(x)=x^3-9x+3 =0.
  $$
  Siguiendo los pasos de la observación~\ref{rk:tema1:separac-ceros},
  calculamos la derivada de $f$,
  $$
  f'(x)=3x^2-9,
  $$
  que sólo se anula en $x=-\sqrt 3,
  x=+\sqrt 3$. Por lo tanto existen, a lo sumo, tres ceros de $f$ que
  están localizados en
  $$
  (-\infty,-\sqrt 3), (-\sqrt 3, +\sqrt 3) \text{ y } (+\sqrt 3,
  +\infty).
  $$

  Solamente resta utilizar el teorema de Bolzano para detectar en
  cuáles de estos intervalos existe una raíz de $f(x)=0$. Como apoyo,
  podemos utilizar un entorno informático para representar la gráfica
  (figura~\ref{fig:tema1:ejemplo-separ-soluc-1}), que en este caso nos
  sugiere que las raíces pueden ser localizadas, concretamente, en los
  intervalos $[-4,-3]\subset (-\infty,-\sqrt 3)$,
  $[0,1]\subset (-\sqrt 3, +\sqrt 3)$ y
  $[2,3]\subset (+\sqrt 3, +\infty)$.
  \begin{figure}
    \begin{graficaTikz}[width=23em, height=15em]
      \begin{axis}[\axisXYmiddle]
        % Draw a curve
        \addplot[domain=-4.0:4.0, blue, ultra thick, samples=40]
        {x^3-9*x+3};
        % Plot a label at curve root
        \node[coordinate, medium dot, pin=95:{$\cero_1$}] at
        (axis cs:-3.15,0) {};
        \node[coordinate, medium dot, pin=85:{$\cero_2$}] at
        (axis cs:0.33,0) {};
        \node[coordinate, medium dot, pin=95:{ $\cero_3$}] at
        (axis cs:2.81,0) {};
      \end{axis}
    \end{graficaTikz}
    \caption{Gráfica de $f(x)=x^3-9x+3$}
    \label{fig:tema1:ejemplo-separ-soluc-1}
  \end{figure}
  Este hecho lo podemos confirmar utilizando el teorema de Bolzano:
  \begin{itemize}
  \item En $[-4,-3]$: $f(-4)=-25$ y $f(-3)=3$, por lo tanto existe (al
    menos) un valor   $\cero_1 \in (-4,3)$ tal que $f(\cero_1)=0$.
  \item En $[0,1]$: $f(0)=3$ y $f(1)=-5$, luego existe
    $\cero_2 \in (0,1)$ tal que $f(\cero_2)=0$.
  \item En $[2,3]$: $f(2)=-7$ y $f(3)=3$, por tanto existe
    $\cero_3 \in (2,3)$ tal que $f(\cero_3)=0$.
  \end{itemize}
\end{example}

\begin{example}
  Determinaremos el número de valores $x\in\Rset$ tales que
  $2x=cos(x)$ y localizaremos estos valores en intervalos (lo que, en
  las próximas secciones, se usará para aplicar métodos numéricos para
  aproximar las soluciones).

  Para ello, planteamos el problema~\eqref{eq:raiz} para la función $f$
  (continua y derivable) definida por
  $$
  f(x)=2x-\cos(x)
  $$
  y, siguiendo las ideas de la
  observación~\ref{rk:tema1:separac-ceros}, comenzamos estudiando la
  derivada
  $$
  f'(x)=2+\sen(x).
  $$
  Puesto que $|\sen(x)|\le 1$, tenemos $f'(x)\neq 0$ para todo
  $x\in\Rset$, luego existe, como máximo, una raíz de $f$ en
  $(-\infty,+\infty)$.

  \begin{figure}
    \begin{graficaTikz}[width=23em, height=15em]
      \begin{axis}[\axisXYmiddle]
        % Draw a curve
        \addplot[domain=-pi:pi+0.3, blue, ultra thick, samples=40]
        {{2*x-cos(deg(x))}};
        % Plot a label at curve root
        \node[coordinate, medium dot, pin=-87:{$\cero$}] at (axis cs:0.45,0) {};
      \end{axis}
    \end{graficaTikz}
    \caption{Gráfica de $f(x)=2x-\cos(x)$}
    \label{fig:tema1:ejemplo-separ-soluc-2}
  \end{figure}
  La gráfica de $f$ (figura~\ref{fig:tema1:ejemplo-separ-soluc-2}) nos
  sugiere que esta raíz, $\alpha$, es positiva. Por ejemplo si
  aplicamos el teorema de bolzano en el intervalo $[0,\pi/2]$ (elegido
  por resultar sencilla la evaluación de $f$) tenemos: $f(0)=-1$ y
  $f(\pi/2)=\pi$.  Así, $x=\alpha$ es el único valor, concretamente
  localizado en $(0,\pi/2)$, tal que $2x=\cos(x)$.
\end{example}

\section{El método de bisección}
\label{sec:tema1:bisecc}

Comenzamos presentando el método más sencillo, basado en el teorema de
Bolzano. Para ello, supongamos que $f$ es una función continua en
$[a,b]$ tal que $f(a)f(b)<0$, por tanto existe algún cero
de $f$ en $(a,b)$, al que llamaremos $\alpha$.

Por simplicidad, supondremos que $\alpha$ es el único cero de $f$ en
$(a,b)$; en otro caso podríamos separar los ceros en subintervalos,
véase la sección~\ref{sec:tema1:exist-y-unic}. Aunque veremos que este
método sigue siendo válido aunque no haya unicidad de solución
(observación~\ref{rk:tema1:unicidad-bisecc}).

La idea del método de bisección es dividir por la mitad el intervalo y
considerar los dos subintervalos resultantes, para seleccionar aquel
en el que $f$ cambia de signo (y que contiene a la raíz, según el
teorema de Bolzano). Con más detalle: el método consiste en construir
una sucesión de intervalos,
\begin{equation}
  \label{eq:tema1:bisecc:0}
  [a,b]=[a_0,b_0] \supset [a_1,b_1] \supset [a_2,b_2] \cdots \supset
  [a_n,b_n] \supset \cdots
\end{equation}
definida a partir de sus puntos medios, $c_n=(a_n+b_n)/2$, tal y como sigue:
\begin{itemize}
\item Como inicialización, tomamos $a_0=a$ y $b_0=b$ y calculamos
  $c_0=(a_0+b_0)/2$.
\item En cada etapa $k\ge 1$, construimos un intervalo $[a_k,b_k]$ a
  partir del anterior, $[a_{k-1}, b_{k-1}]$, de la siguiente forma:
  \begin{enumerate}
  \item Calculamos el punto medio del intervalo anterior,
    $c_{k-1}=(a_{k-1}+b_{k-1})/2$. Si $f(c_{k-1})=0$ hemos terminado
    ($c_{k-1}$ es el cero de $f$).
  \item En otro caso definimos $a_k$ y $b_k$ de forma que $f$ cambie de
    signo en $[a_k,b_k]$:
    \begin{enumerate}
    \item Si $f(a_{k-1})f(c_{k-1})<0$, elegimos $[a_k,b_k]=[a_{k-1}, c_{k-1}]$.
    \item Si $f(c_{k-1})f(b_{k-1})<0$, elegimos $[a_k,b_k]=[c_{k-1}, b_{k-1}]$.
    \end{enumerate}
  \item A continuación, pasamos a la siguiente etapa (incrementamos
    $k$).
  \end{enumerate}
\end{itemize}

En la práctica, es muy improbable que $f(c_k)=0$ y deben establecerse
condiciones para evitar bucles infinitos. Puede verse una
implementación\footnote{Con el fin que las
  implementaciones de los algoritmos sean sencillas, se utilizará el
  lenguaje Python 3.} de bisección en el
Programa~\ref{pro:metodo-biseccion}.

\begin{example}
  Aplicaremos el método de bisección a la función $f(x)=x^3-9x+3$ en el intervalo
  $[0,1]$, donde sabemos que existe un único cero (según vimos en el
  ejemplo~\ref{ex:tema1:separ-soluc-1}). El método consiste en los
  siguientes pasos (resumidos en el cuadro~\ref{tab:tema1:bisecc}):

  \begin{itemize}
  \item Empezamos seleccionando $[a_0,b_0]=[0,1]$.
  \item En la primera etapa, $k=1$, calculamos
    $c_0=(1+0)/2=0.5$  y
    $f(c_0)=f(1/2)=-11/8=-1.375$. Como $f(a_0)=f(0)=3$, elegimos
    $[a_1,b_1]=[a_0,c_0]=[0,0.5]$.
  \item Repetimos el proceso para $k=2,3,...$, obteniendo los resultados que se
    muestran en el cuadro~\ref{tab:tema1:bisecc}.
  \end{itemize}

\end{example}
\begin{table}
  \centering
  \rule{0.99\linewidth}{1.6pt}
  \begin{equation*}
    \begin{array}{l<{\quad}rrrrrr}%{>{$}r<{$}>{$}r<{$}>{$}r<{$}>{$}r<{$}>{$}r<{$}}
      k &  a_k & b_k & c_k & f(a_k) & f(b_k) & f(c_k)
      \\ \toprule%\mbox{}% \mbox{} for avoiding bug with "["
      0 & 0 &1  &  0.5 & 3 & -5 & -1.375
      \\ \noalign{\smallskip}
      1 &  0& 0.5 &  0.25 & 3 & -1.375 & 0.76
      \\ \smallskip
      2 & 0.25 & 0.5 & 0.375 & 0.76 & -1.375 & -0.32
      \\ \smallskip
      3& 0.25 & 0.375 & 0.3125 & 0.76 & -0.32 & 0.21
      \\
      \hfill \vdots \hfill~ & \hfill \vdots \hfill~ &
                                                      \hfill \vdots \hfill~ & \hfill \vdots \hfill~ &
                                                                                                      \hfill \vdots \hfill~ & \hfill \vdots \hfill~
    \end{array}
  \end{equation*}
  \rule{0.99\linewidth}{1.5pt}
  \caption{Método de bisección para $f(x)=x^3-9x-3$ en $[0,1]$.}
  \label{tab:tema1:bisecc}
\end{table}

Como se puede observar en el ejemplo anterior, el método de bisección
genera una sucesión de intervalos $[a_k,b_k]$ que
contienen la solución de~\eqref{eq:raiz} (pues, por construcción,
$f(a_k)f(b_k)<0$). Además, el tamaño $b_k-a_k$ converge a cero (puesto
que, en cada paso, el tamaño se divide por dos), de hecho:
\begin{equation}
  \label{eq:tema1:bisec:1}
  b_k-a_k = \frac{b-a}{2^k} \quad \forall k=0,1,2,...
\end{equation}

Así, se tiene el siguiente resultado:
\begin{theorem}
  \label{thm:tema1:bisecc}
  Sea $f$ una función continua en $[a,b]$ tal que $f(a)f(b)<0$.
  \begin{enumerate}
  \item La sucesión $\{c_k\}_{k=0}^\infty$ (definida por los puntos
    medios de los intervalos en el método de bisección) es convergente
    y su límite, $\cero$, es un cero de $f$ en $(a,b)$.
  \item De hecho, se verifica la siguiente cota del error:
    \begin{equation}
      \label{eq:tema1:bisecc:2}
      |c_k-\cero| < \frac{b-a}{2^{k+1}}.
    \end{equation}
  \end{enumerate}
\end{theorem}

\begin{proof}
  Los intervalos $[a_k,b_k]$ generados por el método de bisección
  verifican~\eqref{eq:tema1:bisecc:0} y $f(a_k)f(b_k)<0$, por lo tanto
  todos ellos contienen algún cero $\alpha$.

  Por definición del método de bisección, en la iteración $k+1$ se
  tiene que $c_k=a_{k+1}$ o $c_k=b_{k+1}$ y que
  $\cero \in (a_{k+1},b_{k+1})$. Teniendo esto en cuenta,
  $$|c_k-\cero|<|b_{k+1}-a_{k+1}|.$$
  Aplicando ahora~(\ref{eq:tema1:bisec:1}), tenemos que
  $|b_{k+1}-a_{k+1}|<(b-a)/2^{k+1}$, de donde
  deduce~\eqref{eq:tema1:bisecc:2}. Y, a su vez,
  \eqref{eq:tema1:bisecc:2} implica que $c_k$ converge a $\cero$.
\end{proof}

\begin{remark}[Ventajas e inconvenientes del método de bisección]
  \label{rk:tema1:unicidad-bisecc}
  ~
  \begin{itemize}
  \item El teorema~\ref{thm:tema1:bisecc} es válido incluso si $f$
    tiene \emph{varios ceros} en $[a,b]$. Esto significa una
    \textbf{ventaja} del método de bisección frente a otras
    alternativas: siempre converge hacia una solución,
    independientemente de que ésta sea única, y sin más hipótesis
    sobre $f$ que su continuidad.

    De hecho, el método de bisección no utiliza ningún tipo de
    información acerca de la función y, solamente requiere la
    evaluación de $f$ en los puntos medios de los intervalos. En la
    práctica, es un método sencillo que puede ser útil como ``método
    de arranque'' (con el que realizar iteraciones previas antes de
    pasar a otros métodos de mayor orden), o como base para métodos
    más eficientes.

  \item Sin embargo, tiene como \textbf{inconveniente} su lenta
    \textbf{velocidad de convergencia}. En concreto, se puede
    demostrar que el método no alcanza el orden $1$ en general
    (aunque sí en algunos casos particulares), necesitando por tanto
    muchas más iteraciones que otros métodos que veremos más tarde.
    Este resultado no es inmediato, se pueden consultar los detalles
    en la bibliografía.

    De hecho, la sucesión que genera el método no garantiza siquiera
    que el error se reduzca en todas las iteraciones (podría ocurrir
    que, en una iteración concreta, $c_k$ sea muy cercano a $\alpha$
    pero $c_{k+1}$, el punto medio del nuevo intervalo, se aleje de
    $\alpha$).


  \end{itemize}
\end{remark}

\begin{remark}
  \label{rk:tema1:bisecc:iteraciones}
  La desigualdad~\eqref{eq:tema1:bisecc:2} nos garantiza que, para $n$
  suficiente grande, podemos aproximar un cero de $f$ con un error tan
  pequeño como deseemos. Y, de hecho, dada una tolerancia
  $\varepsilon$, para que $|c_k-\alpha|<\varepsilon$, es suficiente
  hallar $k$ de forma que el segundo miembro
  de~(\ref{eq:tema1:bisecc:2}) sea menor que $\varepsilon$, es decir,
  realizar iteraciones hasta que $(b-a)/2^{k+1} < \varepsilon$.

  Para ello, planteamos la igualdad $(b-a)/2^{k+1} = \varepsilon$ y, despejando $k$, obtenemos:
  \begin{equation*}
    k=\log_2\left(\frac{b-a}{\varepsilon}\right)-1.
  \end{equation*}
  Como, en general, el resultado de la operación no es un número
  entero, el número de iteraciones que realizaremos, $k_0$, será igual
  al primer entero mayor o igual a la expresión anterior.
\end{remark}

\begin{test}[Función ``bisección'']
  El programa~\ref{pro:metodo-biseccion} muestra una implementación
  del método de bisección. Concretamente, la función
  \pythoninline{biseccion(f, a, b)} calcula una solución
  de~\eqref{eq:raiz} en un intervalo $(a,b)$. A través de parámetros
  opcionales se puede determinar una tolerancia, $\varepsilon$, de
  forma que el ciclo de iteraciones se detenga cuando
  $b_k-a_k<\varepsilon$. La igualdad~(\ref{eq:tema1:bisec:1})
  garantiza que así sera para $k$ suficientemente grande (dado
  concretamente en la
  observación~\ref{rk:tema1:bisecc:iteraciones}). Adicionalmente, la
  función permite fijar el número máximo de iteraciones permitidas y
  seleccionar si se actuará de forma silenciosa o bien con
  verbosidad al mostrar los resultados.

  Por ejemplo, el siguiente código \textit{Python} aproxima un cero de
  la función $f(x)$ introducida en el
  ejemplo~\ref{ex:tema1:separ-soluc-1}
  con tolerancia $10^{-2}$:
  \pythonexternal{tema1/src/biseccion-test1.py}
  \begin{pythonoutput}
    \pythonexternal[backgroundcolor=\color{white},title={Resultado}]{tema1/src/biseccion-test1.out}
  \end{pythonoutput}
\end{test}


\begin{program}
  \widepythonexternal{tema1/src/biseccion.py}
  \caption{Una implementación del método de bisección}
  \label{pro:metodo-biseccion}
\end{program}

\section{Los métodos de punto fijo}
\label{sec:metodos-de-punto-fijo}

En este capítulo introducimos un tipo de métodos que son definidos a
partir de una reformulación de~(\ref{eq:raiz}). Se trata de los
métodos de punto fijo o de aproximaciones sucesivas.
Llamamos \resaltar{punto fijo} de una función continua
$g:I\subset\Rset\to\Rset$ a una solución del siguiente  problema:
\begin{equation}
  \tag{$P_{\text{PF}}$}
  \text{Hallar $\cero\in I$ tal que} \quad \cero=g(\cero).
  \label{eq:punto-fijo}
\end{equation}

Geométricamente, una solución de la ecuación $x=g(x)$ se corresponde
con la abscisa correspondiente a un punto de corte entre la gráfica de
la función $y=g(x)$ y la recta $y=x$ (ver
figura~\ref{fig:ejemplo-punto-fijo-1}).

\begin{example}
  La función $g(x)=2-x^2$ tiene dos puntos fijos en $x=1$ y
  $x=-2$, pues $g(1)=2-1=1$ y $g(-2)=2-4=-2$. No tiene más puntos
  fijos en $\Rset$, pues éstos son raíces de $g(x)-x$, que en este
  caso es un polinomio de grado dos. En la
  figura~\ref{fig:ejemplo-punto-fijo-1} se representan geométricamente
  esta función y sus dos puntos fijos.
\end{example}

\begin{figure}
  \begin{graficaTikz}[width=18em, height=15em]
    \begin{axis}[\axisXYmiddle,
      % legend style = {anchor=north west, pos = north east}]
      legend pos = outer north east, legend cell align=left]
      % Draw a curve
      \addplot[domain=-2.4:2.4, blue, ultra thick, samples=40] {2-x^2};
      \addplot[domain=-2.5:2.5, gray, ultra thick, samples=40] {x};
      % Plot a label at curve root
      \node[coordinate, medium dot, pin=0:{\scriptsize$(1,1)$}]
      at (axis cs:1,1) {};
      \node[coordinate, medium dot, pin=-45:{\scriptsize$(-2,-2)$}]
      at (axis cs:-2,-2) {};
      \legend {$y=g(x)$,$y=x$};
    \end{axis}
  \end{graficaTikz}
  \caption{Puntos fijos de $g(x)=2-x^2$}
  \label{fig:ejemplo-punto-fijo-1}
\end{figure}

Si $\cero$ es una solución de un problema de punto fijo, entonces
$\cero$ es solución de un problema de cálculo de raíces del
tipo~(\ref{eq:raiz}) para $f(x)=x-g(x)$. Recíprocamente un problema de
cálculo de raíces puede ser transformado en un problema del
tipo~(\ref{eq:punto-fijo}) de numerosas formas, por ejemplo
escribiendo $g(x)=x-f(x)$ o empleando otro tipo de manipulaciones
algebraicas (véase el ejemplo~\ref{ex:punto-fijo-1}). Algunas de estas
transformaciones en ecuaciones de tipo punto fijo pueden dar lugar a
potentes técnicas iterativas.

A continuación, estudiamos resultados de existencia y unicidad de
punto fijo.

\begin{proposition}[Existencia de solución de~(\ref{eq:punto-fijo})]
  \label{pro:existencia-punto-fijo}
  Sea $g:[a,b]\to\Rset$ una función continua en $[a,b]$ y supongamos
  que
  \begin{equation}
    g([a,b])\subset [a,b].
    \label{eq:g[a,b].subset.[a,b]}
  \end{equation}
  Entonces existe al menos un punto fijo de $g$ en $[a,b]$.
\end{proposition}
\begin{proof}
  Debido a la hipótesis~(\ref{eq:g[a,b].subset.[a,b]}),
  \begin{extension}
    La hipótesis~(\ref{eq:g[a,b].subset.[a,b]}) se puede debilitar, en
    concreto es suficiente que $g(x)$ verifique
    $$(g(a)-a)(g(b)-b)\le 0$$.
  \end{extension}
  $g(a)\ge a$ y
  $g(b)\le b$, por lo tanto podemos aplicar el teorema de Bolzano
  (teorema~\ref{thm:bolzano}) a la función $f(x)=x-g(x)$ en $[a,b]$.
\end{proof}

\subsection*{Funciones contractivas}

Para estudiar la unicidad de solución de~(\ref{eq:punto-fijo}),
introduciremos la siguiente definición:

\begin{definition}
  Una función $g$ continua en $[a,b]$ se dice \resaltar{contractiva} en
  $[a,b]$ si existe $\cteContract\in[0,1)$ tal que
  \begin{equation*}
    |g(x)-g(y)| \le \cteContract |x-y|, \quad \forall x,y \in [a,b].
    \label{eq:contractividad}
  \end{equation*}
  A $\cteContract$ se le llama constante de contractividad de $g$ en $[a,b]$.
  \label{def:funcion.contractiva}
\end{definition}

\begin{example}
  La función $g(x)=(x^2-1)/3$ es contractiva en $[-1,1]$, pues
  $$
  |g(x)-g(y)|=\left|\frac{x^2-y^2}{3}\right| = \frac{|x+y|}{3}|x-y| \le
  \cteContract |x-y|,
  $$
  para $\cteContract=\max \big\{ \frac{|x+y|}{3}\ /\ x,y\in [-1,1]
  \big\} =\frac23 <1$.
\end{example}

\begin{remark}[Algunas propiedades de las funciones contractivas]~
  \begin{itemize}
  \item Puesto que $\cteContract<1$, toda función contractiva, $g$, <<contrae las
    distancias>> en el sentido de que $|g(x)-g(y)|<|x-y|$ para todo
    $x,y\in [a,b]$.
  \item Toda función contractiva en $[a,b]$ es
    uniformemente continua en ese mismo intervalo (es decir, para
    todo $\varepsilon>0$ existe $\delta>0$ tal que $|x-y|<\delta
    \Rightarrow |f(x)-f(y)|<\varepsilon$).
  \end{itemize}
\end{remark}

\begin{proposition}[Condición suficiente de contractividad]
  \label{pro:1}
  Supongamos que $g\in C^0([a,b])$ y derivable en $(a,b)$, tal que
  \begin{equation}
    \cteContract=\sup_{x\in(a,b)} |g'(x)|<1.
    \label{eq:L=sup|g'|<1}
  \end{equation}
  Entonces $g$ es contractiva en $[a,b]$ y $\cteContract$ es una
  constante de contractividad para $g$.
\end{proposition}
\begin{proof}
  Sean $x,y\in [a,b]$, por ejemplo $x<y$. Aplicando el teorema del
  valor medio en $[x,y]$, deducimos que existe $c\in (x,y)$ tal que
  \begin{equation*}
    |g(x)-g(y)|=|g'(c)(x-y)| \le \cteContract |x-y|,
  \end{equation*}
  donde $\cteContract$ viene dada por~(\ref{eq:L=sup|g'|<1}).
\end{proof}

\begin{remark}
  Si $g'$ es continua en $[a,b]$, es suficiente
  para~(\ref{eq:L=sup|g'|<1}) que $g'(x)<1$ para todo $x\in[a,b]$.
  La demostración: $\sup_{x\in(a,b)}|g'(x)| \le \sup_{x\in[a,b]}
  |g'(x)|$ y, debido al teorema de Weierstrass, $g'$ alcanza su máximo
  en algún punto $c\in [a,b]$.
  \label{rk:3}
\end{remark}
Es sencillo demostrar que las funciones contractivas no pueden tener
más de un punto fijo:

\begin{proposition}[Unicidad de solución de~(\ref{eq:punto-fijo})]
  \label{pro:unicidad-punto-fijo}
  Sea $g:[a,b]\to\Rset$ una función \emph{contractiva} en
  $[a,b]$. Entonces $g$ posee, a lo sumo, un punto fijo en $[a,b]$.
\end{proposition}

\begin{proof}
  Si suponemos que $g$ tiene dos puntos fijos, $\cero_1$ y
  $\cero_2$, llegamos inmediatamente a una contradicción:
  $$
  |\cero_1-\cero_2| = |g(\cero_1)-g(\cero_2)| \le \cteContract |\cero_1 -
  \cero_2| < |\cero_1-\cero_2|,$$
  donde $\cteContract<1$ es la constante de contractividad de $g$.
\end{proof}

\subsection*{Métodos de punto fijo}

Los problemas de punto fijo~(\ref{eq:punto-fijo}) dan lugar al
siguiente tipo de esquemas recursivos, conocidos como métodos de
aproximaciones sucesivas (o simplemente, \resaltar{métodos de punto
  fijo}):
\begin{equation}
  \tag{$M_{\text{AS}}$}
  \left\{
    \begin{array}{l}
      \text{Dado } x_0\in [a,b], \\
      \text{calcular } x_{k+1}=g(x_k), \quad \forall k\ge 0.
    \end{array}
  \right.
  \label{eq:MAS}
\end{equation}
Véase que para que el método esté \textit{bien definido} es necesario
que $x_k$ se encuentre en el dominio de $g$ para todo $k$.
El siguiente teorema resume los resultados de existencia y unicidad
enunciados en las Proposiciones~\ref{pro:existencia-punto-fijo}
y~\ref{pro:unicidad-punto-fijo}, a la vez que garantiza el buen
planteamiento y la convergencia de~(\ref{eq:MAS}) bajo las hipótesis
de aquellas proposiciones.

\begin{theorem}[Teorema del punto fijo de Banach]
  \label{thm:punto-fijo-Banach}
  Sea $g:[a,b]\to\Rset$ tal que $g([a,b]) \subset [a,b]$ y supongamos
  que $g$ es contractiva en $[a,b]$ con constante de contractividad
  $\cteContract\in [0,1)$. Entonces:
  \begin{enumerate}
  \item
    \label{item:punto-fijo-Banach:1}
    La función $g$ tiene un \textsf{único punto fijo}, $\cero$, en
    $[a,b]$.
  \item
    \label{item:punto-fijo-Banach:2}
    Para todo $x_0\in [a,b]$, el método~(\ref{eq:MAS}) está bien
    definido y es \textsf{convergente} hacia $\alpha$.
  \item En cada etapa de~(\ref{eq:MAS}) se tienen las siguientes
    \textsf{estimaciones} del error absoluto:
    \label{item:punto-fijo-Banach:3}
    \begin{align}
      \label{eq:pto-fijo:cota-a-priori}
      |x_k-\cero| &\le \cteContract^k |x_0-\cero|,
      \\
      \label{eq:pto-fijo:cota-a-posteriori}
      |x_k-\cero| &\le \frac{\cteContract}{1-\cteContract}
                    |x_k-x_{k-1}| \le \dots \le
                    \frac{\cteContract^k}{1-\cteContract}
                    |x_1-x_0|.
    \end{align}
  \end{enumerate}
\end{theorem}

La acotación~(\ref{eq:pto-fijo:cota-a-priori}) llama
\textit{estimación a priori}, mientras que
las acotaciones del tipo~(\ref{eq:pto-fijo:cota-a-posteriori}) conocen como
\textit{estimación a posteriori} del error.

\begin{proof}~\par
  El punto~\ref{item:punto-fijo-Banach:1} resulta
  directamente de las Proposiciones~\ref{pro:existencia-punto-fijo}
  y~\ref{pro:unicidad-punto-fijo}. Respecto al
  punto~\ref{item:punto-fijo-Banach:2}, para cualquier $x_0\in [a,b]$,
  el método iterativo está bien definido, ya que para $k>1$,
  $x_{k}=g(x_{k-1})$ y $g([a,b])\subset [a,b]$. Acerca de la
  convergencia, dado $x_0\in [a,b]$ podemos aplicar repetidamente la
  función $g$ obteniendo:
  \begin{equation}
    \begin{aligned}
      |x_k-\cero| = &|g(x_{k-1})-g(\cero)| \le \cteContract
      |x_{k-1}-\cero| = \\
      = \cteContract &|g(x_{k-2})-g(\cero)| \le \cteContract^2
      |x_{k-2}-\cero| \le \cdots \le \cteContract^k |x_0-\cero|.
    \end{aligned}\label{eq:1}
  \end{equation}
  Así se tiene la estimación a
  priori~(\ref{eq:pto-fijo:cota-a-priori}) y, como
  $\lambda<1$, podemos concluir que $x_k\to\cero$.
  %
  La demostración la estimación~(\ref{eq:pto-fijo:cota-a-posteriori})
  es algo más técnica y se relega a una nota a pie de página\footnote{
    Para probar~(\ref{eq:pto-fijo:cota-a-posteriori}) comenzamos con
    un razonamiento análogo al anterior. Utilizando que $g$ es
    contractiva (y que $x_k\in [a,b]$ para todo $k\ge 0$) tenemos que
    para todo $n \ge 0$:
    \begin{equation*}
      |x_{n+1}-x_{n}| =
      |g(x_{n})-g(x_{n-1})|\le\cteContract|x_{n}-x_{n-1}| \le \cdots \le \cteContract^n|x_1-x_0|.
    \end{equation*}
    Sean $k,n\in\Nset$, con $n>k$. Lo anterior junto a la
    desigualdad triangular implican:
    \begin{align*}
      |x_n-x_k| &\le |x_n-x_{n-1}| + |x_{n-1}-x_{n-2}| + \cdots +
                  |x_{k+2}-x_{k+1}| + |x_{k+1}-x_{k}|
      \\
                &\le \left(\cteContract^{n-k} + \cteContract^{n-k+1} +\cdots+
                  \cteContract^{2} + \cteContract \right) |x_k-x_{k-1}|
      \\
                &=\left(\frac{\cteContract-\cteContract^{n-k+1}}{1-\cteContract}\right)
                  |x_{k}-x_{k-1}|.
    \end{align*}
    Tomando $n\to+\infty$ en los dos miembros de la desigualdad
    anterior, $x_n\to\alpha$ y $\lambda^{n-k+1}\to 0$, así:
    $$
    |x_k-\cero| \le \frac{\cteContract}{1-\cteContract}
    |x_{k}-x_{k-1}|.
    $$
    Y aplicando reiteradamente la contractividad, se obtienen las
    demás desigualdades de~(\ref{eq:pto-fijo:cota-a-posteriori}).
  }.
\end{proof}

\begin{algorithm} \begin{python}
    def punto_fijo(g, x0, tol, max_iters):
    iter = 0
    while iter < max_iters:
    x1 = g(x0)
    if abs(x1-x0) < tol: return x1
    x0 = x1
    iter = iter + 1
    print "Fallo de convergencia en el método de punto fijo!"
  \end{python}
  \caption{Método de punto fijo (o de las aproximaciones sucesivas)}
  \label{alg:metodo-punto_fijo}
\end{algorithm}

\begin{remark}
  En la práctica, la estimación a
  posteriori~(\ref{eq:pto-fijo:cota-a-posteriori}) garantiza la
  acotación del error de truncamiento, de forma que éste se hará
  pequeño siempre que realicemos iteraciones de~(\ref{eq:MAS}) hasta
  que el error entre dos iteraciones consecutivas sea pequeño (en
  concreto, si $|x_{k+1}-x_k|<\delta$, donde $\delta$ es una
  tolerancia prefijada, entonces
  $|x_{k+1}-\alpha|<\epsilon=\delta\cdot\lambda/(1-\lambda)$). Véase al
  algoritmo~\ref{alg:metodo-punto_fijo}.
  \label{rk:pto-fijo:estimacion-error}
\end{remark}

\begin{example}
  Nos proponemos el formular en términos de problemas de punto fijo el
  cálculo de todos los ceros de $f(x)=\sen(x)-4x^2+1$.


  \textsf{Primera etapa:} Determinar cuántos ceros son y en qué
  intervalos se encuentran. Como $f$ es continua y derivable en todo
  $\Rset$, utilizaremos el algoritmo de separación de ceros enunciado
  en la observación~\ref{rk:tema1:separac-ceros}.
  \begin{enumerate}
  \item Debemos primero localizar los ceros de la primera
    derivada, planteando $f'(x)=\cos(x)-8x=0$, problema cuya solución
    no es inmediata.
  \item Aun así, aprovecharemos que $f'(x)$ es a su vez continua y
    derivable, con $f''(x)=-\sin(x)-8$. Como $f''(x)<0$ para todo
    $x\in\Rset$, el corolario~\ref{cor:tema1:exist+unic} (aplicado a
    $f'$) implica que $f'$ tiene un único cero en $\Rset$.
  \item Usando de nuevo el
    corolario~\ref{cor:tema1:exist+unic}, aplicado en esta ocasión a
    $f$, concluimos que $f$ tiene exactamente dos ceros en $\Rset$
    (separados por el cero de $f'$).
  \item Por último, el teorema de Bolzano nos permitirá determinar dos
    intervalos $[a_1,b_1]$ y $[a_2,b_2]$ que contengan estos dos
    ceros. Podemos proceder al análisis de las regiones donde $f$ es
    positiva y negativa, o bien utilizar un programa de ordenador para
    representar la gráfica como ayuda. En cualquier caso, tendremos
    que probar con distintos intervalos hasta encontrar aquellos en
    los que se verifique el Teorema de Bolzano (y más adelante, las
    hipótesis del Teorema de punto fijo). En este caso, una primera
    elección, válida para el Teorema de Bolzano es:
    \begin{itemize}
    \item $[a_1,b_1]=[0, \pi/2]$, pues $f(0)=1>0$ y $f(\pi/2)= 1-\pi^2+1<0$
    \item $[a_l,b_2]=[-\pi/2, 0]$, pues $f(-\pi/2)=-1-\pi^2+1<0$ y
      $f(0)>0$,
    \end{itemize}
    % El problema es que la función de punto fijo $g_1$ que definiremos
    % más abajo no es derivable en $x=-\pi/2$. Por ello, restringiremos
    % este intervalo, definiéndolo tal y como sigue:
    % \begin{itemize}
    % \item $[a_1,b_1]=[-\pi/4, 0]$. Seguimos estando en las hipótesis
    %   del Teorema de Bolzano, pues $f(-\pi/4)=
    %   -\sqrt{2}/2-\pi^2+1=--9.5767...<0$ y $f(0)=1>0$,
    % \end{itemize}
  \end{enumerate}

  \textsf{Segunda etapa}: intentamos reescribir, en cada uno de estos
  intervalos, el problema en términos de un esquema de punto fijo que
  esté en las hipótesis del teorema~\ref{thm:punto-fijo-Banach}. Para
  ello, podemos proceder de muchas formas, por ejemplo tomar
  $g(x)=f(x)+x$ o bien intentar despejar $x $ en la ecuación,
  $\sen(x)-4x^2+1=0$. En este caso, optamos por esta última
  posibilidad, llegando a $x=\pm \frac 12 \sqrt{1+\sen(x)}$.
  \begin{itemize}
  \item En $[a_1,b_1]=[0, \pi/2]$, definiremos el
    problema de punto fijo
    $$
    x=g_1(x)=+\dfrac 12 \sqrt{1+\sen(x)}.
    $$
    La función $g_1$ está bien definida (pues $1+\sen(x)\ge 0$) y es
    continua en todo $\Rset$. Veamos que está en las hipótesis del
    teorema de punto fijo. En primer lugar,
    \begin{equation}
      g_1'(x)=\frac{\cos(x)}{4\sqrt{1+\sen(x)}},
      \label{eq:6}
    \end{equation}
    que está bien definida para todo $x\in[0, \pi/2]$, ya que
    $\sen(x)\ge 0$ en este intervalo. Además:
    \begin{enumerate}
    \item $g_1([0,\pi/2])\subset [0,\pi/2]$:
      \begin{itemize}
      \item $g_1(0)=1/2 \in [0, \pi/2],$
      \item $g_1(\pi/2)= \frac12 \sqrt{1+\sen(\pi/2)}=\frac{\sqrt 2}2 \in
        [0, \pi/2]$ (pues $\sqrt 2 < 2 < \pi$)

      \item $g_1$ es monótona (en concreto, creciente) en $[0,\pi/2]$:
        En efecto, como $\cos(x)\ge 0$ si $x\in
        [0, \pi/2]$, $g_1'(x)\ge 0$ en este intervalo (ver~(\ref{eq:6})).
      \end{itemize}
      Por lo tanto,
      $g([0, \pi/2]) \subset [0, \pi/2]$.
    \item Veamos que $g_1$ es contractiva en $[0,\pi/2]$, para lo que
      es suficiente que
      $$|g'(x)|<1 \forall x\in [0, \pi/2]$$
      (ver la
      observación~\ref{rk:3}). Como $\sen(x)$ es creciente en
      $[0, \pi/2]$, en este intervalo se verifica que
      $$4\sqrt{1+\sen(x)}\ge 4\sqrt{1+\sen(0)}=4,$$
      así
      $$
      \left|g'(x)\right| = \frac{|\cos{x}|}{4\sqrt{1+\sen(x)}} \le
      \frac{1}{4\sqrt{\sen(x)+1}} \le \frac 1 4 < 1.
      $$
      En concreto, como constante de contractividad podemos escoger
      $$\lambda=\frac 1 4.$$
    \end{enumerate}
    % \item En $[a_1,b_1]=[-\pi/4, 0]$, como $x<0$, definiremos el
    %   problema de punto fijo
    %   $$
    %   x=g_1(x)=-\dfrac 12 \sqrt{1+\sen(x)}.
    %   $$
    %   La función $g_1$ está bien definida (pues $1+\sen(x)\ge 0$) y es
    %   continua en todo $\Rset$. Veamos que está en las hipótesis del
    %   teorema de punto fijo. En primer lugar,
    %   $$
    %   g_1'(x)=\frac{-\cos(x)}{4\sqrt{1+\sen(x)}},
    %   $$
    %   que está bien definida para todo $x\in[-\pi/4, 0]$, ya que
    %   $\sen(x)>-1$ en este intervalo. Además:
    %   \begin{enumerate}
    %   \item $g_1([a_1,b_1])\subset [a_1,b_1]$: En efecto, $g_1$ es
    %     decreciente en $[-\pi/4,0]$ (pues $\cos(x)\ge 0$ si $x\in
    %     [-\pi/4, 0]$ y por tanto $g_1'(x)\le 0$ en este intervalo).
    %     Como
    %     \begin{align*}
    %       g_1(-\pi/4)&= -\frac12 \sqrt{1+\sen(-\pi/4)}=-0.270...\in
    %       [-\pi/4,0]=[-0.785...,0],
    %       \\
    %       g_1(0)&=-1/2 \in [-\pi/4,0],
    %     \end{align*}
    %     se tiene que $g([-\pi/4,0])\subset
    %     [g(0),g(-\pi/4)] \subset [-\pi/4,0]$.
    %   \item Veamos que $g_1$ es contractiva en $[a_1,b_1]$, para lo que es
    %     suficiente que $|g'(x)|<1$ en $[a_1,b_1]=[-\pi/4,0]$ (según la
    %     proposición~\ref{pro:1} y a la observación~\ref{rk:3}).  Como
    %     $\sen(x)$ es creciente en $[-\pi/4,0]$, en este intervalo se
    %     verifica que
    %     $$4\sqrt{1+\sen(x)}\ge 4\sqrt{1+\sen(-\pi/4)}=2.164... >1,$$
    %     así
    %     $$
    %     \left|g'(x)\right| = \frac{|\cos{x}|}{4\sqrt{1+\sen(x)}} \le
    %     \frac{1}{4\sqrt{\sen(x)+1}} < 1.
    %     $$
    %   \end{enumerate}
  \item En $[a_2,b_2]=[-\pi/2,0]$ podemos proceder de forma similar,
    tomando $$g_2(x)=-\frac 12 \sqrt{1+\sen(x)}$$
    (necesitamos el signo
    negativo porque queremos que $g([-\pi/2,0])\subset\Rset^{-}$).

    Pero en el intervalo $[-\pi/2,0]$ tenemos una dificultad
    adicional, que aparecerá al considerar la derivada en este
    intervalo:
    $$
    g_2'(x)=\frac{-\cos(x)}{4\sqrt{1+\sen(x)}},
    $$
    En efecto, como $\sen(-\pi/2)=-1$, el denominador se hace cero en
    $[-\pi/2,0]$, por lo tanto la derivada no está bien definida.

    Este problema se resuelve fácilmente: basta considerar cualquier
    intervalo cuyo extremo izquierdo sea mayor que $-\pi/2$. Por
    ejemplo, redefinimos el intervalo como
    $$[a_2,b_2]=[-\pi/4,0].$$
    Procediendo como antes, demostraríamos que
    $g_2([-\pi/4,0])\subset[-\pi/4,0]$ y $g_2$ es contractiva en este
    intervalo, por lo que $g_2$ está en las hipótesis del Teorema del
    punto fijo en $[a_2,b_2]$. El desarrollo de este apartado se deja
    como ejercicio al lector.
  \end{itemize}
  La \textsf{tercera y última etapa} consistiría en
  aplicar~(\ref{eq:MAS}) para aproximar las soluciones en cada uno de
  los dos intervalos. Por ejemplo, el siguiente código muestra el
  resultado de aplicar el programa~\ref{pro:metodo-puntofijo}
  a la función $g_1$ anterior (partiendo de $x_0=1$):
  \pythonexternal{tema1/src/puntofijo-test1.py}
  \begin{pythonoutput}
    \pythonexternal[backgroundcolor=\color{white},title={Resultado}]{tema1/src/puntofijo-test1.out}
  \end{pythonoutput}
  \label{ex:punto-fijo-1}
  \begin{program}
    \widepythonexternal{tema1/src/puntofijo.py}
    \caption{Una implementación en lenguaje Python del método de
      aproximaciones sucesivas para el cálculo un punto fijo $x=g(x)$}
    \label{pro:metodo-puntofijo}
  \end{program}

  Si tenemos en cuenta que, como vimos anteriormente, podemos tomar
  $\lambda=1/4$ como constante de contractividad, podemos usar la
  estimación a posteriori para acotar el error. En efecto, tomando
  $k=6$ en la primera desigualdad
  de~(\ref{eq:pto-fijo:cota-a-posteriori}) y usando el hecho que
  $$
  |x_{6}-x_{5}| \simeq 6.12623522578 \cdot 10^{-7},
  $$
  llegamos a que el error para la aproximación $x_6=1.40962396703$ verifica:
  $$
  |x_6-\cero| \le \frac{\cteContract}{1-\cteContract}
  |x_6-x_{5}|
  \simeq \frac{1}{3}\; 6.12623522578 \cdot 10^{-7}.
  $$
  Por último: utilizando las estimaciones a
  posterior~(\ref{eq:pto-fijo:cota-a-posteriori}) podríamos responder,
  realizando una sola iteración, a la siguiente pregunta: ¿cuántas
  iteraciones son necesarias para garantizar que el error sea menor que
  $10^{-15}$? El desarrollo de la respuesta queda como ejercicio
  propuesto.
\end{example}


\begin{remark}[Orden uno de~(\ref{eq:MAS})]
  En las hipótesis del Teorema~\ref{thm:punto-fijo-Banach} los métodos
  de punto fijo tienen \resaltar{orden de convergencia (al menos)
    uno}. En efecto, razonando de manera similar a la demostración del
  teorema anterior:
  $$
  |x_{k+1}-\cero|=|g(x_{k})-g(\cero)|\le\cteContract |x_{k}-\cero|.
  $$
  Y según la definición~\ref{def:orden-convergencia-al-menos-p}, esta
  desigualdad significa que el método de punto fijo tiene orden (al
  menos) $p=1$ (con constante $\lambda<1$).
\end{remark}
\begin{remark}[Orden $p$ de~(\ref{eq:MAS})]
  Para hipótesis más restrictivas sobre $g$ se puede llegar a
  \resaltar{orden mayor que uno}. En concreto, se puede demostrar
  que si~(\ref{eq:MAS}) converge a un punto fijo, $\cero$, y si
  $g\in C^p([a,b])$ siendo $p \ge 1$ un entero, con
  \begin{extension}
    (En el caso $p=1$ es necesario exigir $g'(\cero)<1$).
  \end{extension}
  \begin{equation*}
    g'(\cero)=g''(\cero)=\cdots=g^{p-1)}(\cero)=0, \quad
    g^{p)}(\cero)\neq 0,
  \end{equation*}
  entonces~(\ref{eq:MAS}) tiene orden exactamente $p$.
  \begin{extension}
    (Siempre que no ocurra que $x_k=\alpha\ \forall k\ge k_0$).
  \end{extension}
  La demostración se basa en un desarrollo de Taylor de $g$ hasta
  orden $p$.
  \label{rk:MAS.orden.p}
\end{remark}
%     % \subsection{Ejemplos}
Para terminar, mostramos un resultado que garantiza la convergencia
de~(\ref{eq:MAS}) hacia un punto fijo conocido, sin utilizar hipótesis
globales sobre el intervalo. En concreto, en el siguiente teorema
solamente se utilizan hipótesis locales sobre la derivada de $g$.

\begin{theorem}[Convergencia local de los métodos de punto fijo]
  \label{thm:punto-fijo-convergencia-local}
  Sea $\cero$ un punto fijo de una función $g$. Supongamos que existe
  $\delta>0$ tal que $g\in C^1(\cero-\delta,\cero+\delta)$ y que
  $|g'(\cero)|<1$. Entonces:
  \begin{enumerate}
  \item Existe $\rho\in (0,\delta)$ tal que método de punto
    fijo~(\ref{eq:MAS}) está bien definido y converge hacia $\cero$
    (para cualquier inicialización $x_0 \in [x_0-\rho,x_0+\rho]$).
  \item Existe una constante de contractividad $\lambda\in [0,1)$ (que
    depende de $\rho$ y $|g'(\cero)|$) para la que se verifican las
    estimaciones a priori~(\ref{eq:pto-fijo:cota-a-priori}) y a
    posteriori(\ref{eq:pto-fijo:cota-a-posteriori}) en
    $[x_0-\rho,x_0+\rho]$.
  \end{enumerate}
\end{theorem}
\begin{proof}
  Simplemente tenemos que aplicar el
  teorema~\ref{thm:punto-fijo-Banach} (Teorema del punto fijo) en un
  intervalo adecuado $[x_0-\rho,x_0+\rho]$. En concreto, como
  $g'(\cero)<1$ y $g \in C^1(\cero-\delta,\cero+\delta)$, podemos
  asegurar que para algún $\rho\in(0,\delta)$,
  $$
  |g'(x)|<1 \quad \forall x\in [\cero-\rho, \cero+\rho],
  $$
  luego $g$ es contractiva en $[\cero-\rho, \cero+\rho]$ (debido a
  la proposición~\ref{pro:1} y a la observación~\ref{rk:3}), con
  constante $\lambda<1$.

  Además es fácil ver que $g([\cero-\rho, \cero+\rho]) \subset
  [\cero-\rho, \cero+\rho]$, pues si $x\in[\cero-\rho, \cero+\rho]$
  se tiene $ |x-\cero|<\rho$ y por tanto
  $$
  |g(x)-\cero| = |g(x)-g(\cero)| \le \lambda |x-\cero|< \rho
  \Rightarrow g(x)\in [\cero-\rho, \cero+\rho].
  $$
\end{proof}

%%% Local Variables:
%%% mode: latex
%%% TeX-master: "../tema1.tex"
%%% End:
