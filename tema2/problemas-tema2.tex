\documentclass[11pt]{article} 
\usepackage[spanish]{babel}
\usepackage[utf8]{inputenc}

% ------ Cargar estilo específico para la relación de problemas
\usepackage{problemas-MNII}

% ------ Para incluir ficheros pdf
\usepackage{pdfpages}

% ==============
\begin{document}
% =============
\begin{flushright}
  \LARGE\it Relación de problemas. Tema \huge 2.\\
  \bigskip
\end{flushright}

\begin{problemas}

  \begin{problema}
    Consideremos la función $f(x)=\sin(x)$ y su polinomio de
    interpolación de Lagrange, $p_n(x)$, en el soporte de puntos
    $S=\{x_k\}_{k=0}^n= \{\pi\,(k+1)/2\}_{i=0}^n$.
    \begin{enumerate}
    \item Utilizando la fórmula de diferencias divididas de Newton,
      calcular $p_3(x)$.
    \item Argumentar si podemos esperar que se verifique 
      $$
      \lim_{n\to\infty} p_n(0) = f(0).
      $$    
      \begin{quote}\em\small
      Indicación: la expresión del error de interpolación de Lagrange es:<
      \begin{equation*}
        f(x)-p_n(x)=\frac{f^{n+1)}(\xi_x)}{(n+1)!}
        (x-x_0)(x-x_1)\dots(x-x_n), \quad\text{con } \xi\in (a,b).
      \end{equation*}
    \end{quote}
  \end{enumerate}
  \end{problema}
  
  \begin{problema}
    Para cada $n\in\Nset$ se considera el soporte definido por $n+1$
    puntos equiespaciados, $S_{n+1}=\{x_0,x_1,...,x_n\}$, en el intervalo
    $[-1,1]$.  Calcular una cota para el error cometido al aproximar
    las siguientes funciones mediante el polinomio de interpolación
    de Hermite, $p_{2n+1}$, en $S_{n+1}$. Analizar, en cada caso, su
    convergencia cuando $n\to\infty$.
    \begin{itemize}
    \item $\displaystyle f_1(x)=\cos x$
    \item $\displaystyle f_2(x)=e^{2x}$
    \end{itemize}
    \begin{quote}\em\small
      Indicación: Si $f\in C^{2n+2}([a,b])$, se tiene la siguiente
      expresión del error de interpolación de Hermite: 
      \begin{equation*}
        f(x)-P(x)=\frac{f^{2n+2)}(\xi_x)}{(2n+2)!}
        (x-x_0)^2(x-x_1)^2\dots(x-x_n)^2, \quad\text{con } \xi\in (a,b).
      \end{equation*}
    \end{quote}
  \end{problema}
  
  \begin{problema}
    Calcular el polinomio de interpolación de Lagrange en los
    siguientes puntos:
    $$
    (0,-5),\ (1,-3),\ (2,1),\ (3,13).
    $$
    \begin{enumerate}
    \item Mediante resolución de un sistema de ecuaciones
    \item Mediante la fórmula de Lagrange
    \item Mediante la fórmula de diferencias divididas de Newton
    \end{enumerate}
    
  \end{problema}
  

  % \begin{problema}
  %   Calcular el polinomio de interpolación de Hermite para la función
  %   $f(x)=\log(x)$ en el soporte $\{1,2\}$. Supuesto conocido el valor
  %     de $\log(2)$, aproximar el valor de $\log(1.5)$, acotando el
  %     error cometido.
  % \end{problema}
  

  
  \begin{problema}
    Sea $p_n(x)$ el polinomio de interpolación de $f(x)=e^x$ en el
    soporte $\{0,1/n,2/n,\dots,(n-1)/n,1\}$. Hallar el valor mínimo
    que debe tener $n$ para poder asegurar que 
    $$
    ||f-p_n||_\infty<10^{-6}.
    $$
    \begin{quotation}
      \small\em Indicación: acotar de la mejor forma posible el
      producto $(x-x_0)(x-x_1)\cdots(x-x_n)$ en el término de error.
    \end{quotation}
  \end{problema}
  
  \begin{problema}
    Sea $f(x)=\log_4(x)$, $x>0$. Calcular $p(32)$, siendo $p$ el
    polinomio de menor grado posible que coincide con $f$ en los
    siguientes puntos:
    \begin{equation*}
      (a) \quad \{x_0,x_1\}=\{1,64\}. \qquad\qquad
      (b) \quad \{x_0,x_1,x_2\}=\{1,64,256\}. 
      \quad
    \end{equation*}
    Calcular, en cada caso, la diferencia $f(32)-p(32)$. Este ejemplo
    muestra que la precisión no mejora, necesariamente, aumentando el
    número de puntos de interpolación.
  \end{problema}
  \end{problemas}
  
  \rule{0.6\linewidth}{2pt}\par
  La página siguiente se corresponde con una bibliografía externa
  [Doubova-Guillen]:
  
  \includepdf[pages={1}]{Pagina-ejercicios-DoubovaGuillen.pdf}
\end{document}

%%% Local Variables: 
%%% mode: latex
%%% TeX-master: t
%%% End: 
