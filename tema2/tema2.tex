
\chapter[Raíces de funciones de una variable]{Interpolación y aproximación de funciones%
  \footnote{\licenseInfo}}
\label{cha:Interpolacion-aproximacion}

El problema de la interpolación surge cuando se intenta construir una
función (a la que suele llamarse función interpolante) de la que
solamente conocemos sus valores en una serie de puntos (que suelen
llamarse nodos de interpolación).  Este tipo de problemas surge, por
ejemplo, si disponemos de un cierto número de datos puntos obtenidos
por muestreo o a partir de un experimento y pretendemos construir una
función que los ajuste, o bien si pretendemos aumentar el tamaño de
una fotografía, rellenando la imagen inicial con datos que son
<<inventados>>.

Un problema estrechamente ligado con el de la interpolación es la
aproximación de una función complicada por una más simple y por tanto
más fácil de manejar (derivar, integrar, etc). Por ejemplo, los
métodos usuales para la aproximación de la integral de una función en
un intervalo se basan en sustituirla por una función polinómica.
En lo que sigue, supondremos que la función interpolante es un
polinomio (aunque existen otras posibilidades: interpolación por
funciones racionales, trigonométricas, exponenciales, etc.), puesto
que son las funciones más sencillas de manejar y constituyen la
base de los métodos numéricos que estudiaremos en los próximos capítulos. 

\section{Interpolación de Lagrange}
\label{sec:interp-de-lagrange}

Consideremos conjunto de datos definido por un soporte de $n+1$ nodos
distintos, $S=\{x_0,x_1,\dots,x_n\}\subset \Rset$, junto a $n+1$
puntos $y_i \in \Rset$:
\begin{equation}
  \begin{array}{r|lllcl}
    \toprule
    x & x_0 & x_1 & x_2 & \dots & x_n
    \\ \midrule
    y & y_0 & y_1 & y_2 & \dots & y_n
    \\
    \bottomrule
  \end{array}
  \label{eq:tabla-datos-lagrange}
\end{equation}
y denotemos por $\Pol_n[x]$ al conjunto de los polinomios de grado
menor o igual a $n$ en la variable $x$. Planteamos el problema de la
interpolación de Lagrange como:
\begin{equation}
  \tag{P$_{IL}$}
  \text{Hallar } p\in\Pol_n[x] \tq p(x_i)=y_i \quad \forall i=0,1,\dots,n.
  \label{eq:problema-interpol-lagrange}
\end{equation}
\begin{definition}
  \label{def:interpolador-lagrange}
  Decimos que un polinomio $p$ interpola el conjunto de
  datos~\eqref{eq:tabla-datos-lagrange} si $p$ es solución del
  problema~\eqref{eq:problema-interpol-lagrange}. En el caso en que
  los datos vengan dados por una función, $y_i=f(x_i)$ decimos que $p$
  es el polinomio de interpolación de Lagrange (o interpolador de
  Lagrange) de la función $f$ en los nodos $x_0$, $x_1$, \dots, $x_n$.
\end{definition} 
En adelante supondremos $f\in C^0([a,b])$ para cierto intervalo
$[a,b]\subset\Rset$ que contiene a los nodos de interpolación,
$x_0,x_1,\dots,x_n \in [a,b]$.  El resto de esta sección se estructura
de la siguiente forma:
\begin{enumerate}
\item Estudio de la existencia y unicidad de solución del
  problema~\eqref{eq:problema-interpol-lagrange}.
\item Deducción de algoritmos para el cálculo efectivo del polinomio
  de interpolación.
\item Análisis del error de interpolación, es decir de la diferencia
  $|f(x)-p(x)|$ con $x\in [a,b]$.
\end{enumerate}

\subsection{Existencia y unicidad del interpolador de Lagrange}
\label{sec:exist-y-unic-lagrange}

\begin{theorem}
  \label{thm:existencia-unicidad-lagrange}
  Dado un soporte de $n+1$ nodos distintos $x_0$, $x_1$,\dots, $x_n
  \in \Rset$ y dados $y_0$, $y_1$,\dots, $y_n\in\Rset$ cualesquiera,
  existe una única solución de~\eqref{eq:problema-interpol-lagrange},
  es decir existe un único polinomio $p\in\Pol_n[x]$ tal que
  $p(x_i)=y_i$ para todo $i=0,1,\dots,n$.
\end{theorem}
\begin{proof}~
\etapa{Etapa a) Equivalencia con un sistema lineal de ecuaciones.}
  Todo polinomio $p\in\Pol_n[x]$, está unívocamente determinado por
  $n+1$ coeficientes, $a_0$, $a_1$, \dots, $a_n\in\Rset$ tales que
  \begin{equation}
    p(x)=a_0 + a_1 x + \cdots + a_n x^n.
  \end{equation}
  A su vez, el polinomio de interpolación se caracteriza por verificar
  $n+1$ ecuaciones,
  $$
  p(x_i)=y_i, \quad i=0,1,\dots,n,
  $$
  que se pueden escribir como el siguiente sistema de $n+1$ ecuaciones
  con $n+1$ incógnitas (cuya matriz cuadrada, $A$, se conoce como
  matriz de Vandermonde):
  \begin{equation}
    \begin{pmatrix}
      1 & x_0& \cdots & x_0^n \\
      1 & x_1& \cdots & x_1^n \\
      \vdots & \vdots & & \vdots \\
      1 & x_n& \cdots & x_n^n 
    \end{pmatrix}
    \begin{pmatrix}
      a_0 \\ a_1 \\ \vdots \\ a_n
    \end{pmatrix}
    =
    \begin{pmatrix}
      y_0 \\ y_1 \\ \vdots \\ y_n
    \end{pmatrix}.
  \end{equation}
  Por tanto la existencia y unicidad del polinomio de interpolación es
  equivalente a la existencia de unos únicos coeficientes
  $(a_0,a_1,...,a_n)$ que solucionan el sistema anterior.
  
  \etapa{Etapa b) Existencia y unicidad de solución.}  Veremos que
  este sistema tiene una única solución. Para ello, como $A$ es una
  matriz cuadrada, basta ver que $|A|\neq 0$, para lo que es suficiente
  probar que el sistema homogéneo $Au=0$, con $u\in\Rset^{n+1}$, tiene
  como única solución $u=0$.

  Supongamos que $u=(a_0,a_1,\dots,a_n) \in\Rset^{n+1}$ verifica
  $Au=0$. Entonces, el polinomio asociado, $p(x)=a_0 + a_1x + \cdots
  + a_n x^n$, verifica $p(x_i)=0$ para $i=0,1,\dots,n$.  Por lo tanto $p$ es un
  polinomio de grado $n$ con $n+1$ raíces distintas, luego
  $p=0$, es decir $u=(a_0,a_1,\dots,a_n)=0$. 
\end{proof}

\begin{remark}
  \label{rk:1}
  En la demostración anterior podríamos haber probado directamente
  que el determinante de la matriz de Vandermonde es distinto de cero,
  pues los puntos $x_0$, $x_1$, \dots, $x_n$ son distintos
  entre sí. Pero se ha optado por realizar la etapa b) debido a que el
  argumento empleado (es decir, en un sistema lineal con matriz
  cuadrada, la unicidad [o sea $Au=0 \Rightarrow u=0$] implica la existencia)
  es generalizable a otras demostraciones de existencia y unicidad que
  aparecerán, por ejemplo, en la siguiente sección.
\end{remark}

\subsection{Construcción del polinomio de interpolación de Lagrange}
\label{sec:construcion--polinomio-lagrange}

De la demostración del Teorema~\ref{thm:existencia-unicidad-lagrange}
se puede deducir un primer método para la construcción del
interpolador de Lagrange: la resolución del sistema lineal con matriz
de Vandermonde. Sin embargo, este método es costoso, pues se trata de
una matriz llena (con pocos ceros) y además se puede demostrar que
esta matriz tiene un mal condicionamiento cuando $n$ crece hacia
infinito. Estudiaremos dos métodos más adecuados.

\subsubsection{Fórmula de Lagrange}
Dado un soporte de $n+1$ puntos distintos, $S=\{x_0,x_1,\dots,x_n\}$,
existe un único polinomio $L_i\in \Pol_n[x]$, que interpola los
valores $y_0=0$, \dots, $y_{i-1}=0$, $y_i=1$, $y_{i+1}=0$, \dots,
$y_n=0$ (debido al Teorema~\ref{thm:existencia-unicidad-lagrange}). Es
decir, $L_i$ verifica (para cada $x_j$):
\begin{equation}
  L_i(x_j)= \delta_{ij} = 
  \left\{\begin{array}{l}
      1 \text{ si } i=j, \\\noalign{\medskip} 0 \text{ si } i\neq j
    \end{array}\right. \quad \text{(función delta de Kronecker).}
\end{equation}
En concreto, es fácil comprobar que el polinomio $L_i(x)$ viene dado
por el siguiente polinomio (de orden exactamente igual a $n$):
\begin{equation}
  L_i(x)=
  \begin{array}{c@{}c@{}c@{}c@{}c@{}c@{}c}
    (x-x_0) & \cdots & (x-x_{i-1}) & \cdot & (x-x_{i+1}) & \cdots &
    (x-x_n)
  \\ \hline
    (x_i-x_0) & \cdots & (x_i-x_{i-1}) & \cdot & (x_i-x_{i+1}) & \cdots &
    (x_i-x_n)
  \end{array}
%   \frac{x-x_0}{x_i-x_0}\cdots
%   \frac{x-x_{i-1}}{x_i-x_{i-1}}\cdot
%   \frac{x-x_{i+1}}{x_i-x_{i+1}}\cdots
%   \frac{x-x_n}{x_i-x_n}.
\end{equation}
El conjunto de los polinomios $\{ L_0, L_1,\dots, L_n \}$ se llama
base de Lagrange asociada al soporte $S=\{x_0,x_1,\dots,x_n\}$.  Es
fácil demostrar que $\{L_i\}_{i=0}^n$ constituyen una base del espacio
vectorial formado por los polinomios de grado exactamente igual a
$n$.

Así, dado un conjunto de valores $\{y_0,y_1,\dots,y_n\}$, el polinomio
que los interpola en el soporte $S$ viene dado por:
\begin{equation}
  p(x)= y_0L_0(x) + y_1 L_1(x) + \cdots + y_n L_n(x),
\end{equation}
pues es evidente que, por construcción de los polinomios $L_i$, 
$p_n$ es solución de~(\ref{eq:problema-interpol-lagrange}).

\begin{remark}
  Una ventaja de la fórmula de Lagrange es que, fijado un soporte $S$
  y una vez calculados los polinomios $\{L_0,L_1,\dots, L_n\}$ que
  forman a la base de Lagrange asociada, es muy sencillo calcular los
  polinomios que interpolan tantos conjuntos de valores,
  $\{y_0,y_1,\dots,y_n\}$ como deseemos. 
\end{remark}

\begin{example}
  \label{sec:formula-de-lagrange}
  Calculemos el polinomio de interpolación de Lagrange asociado a
  los puntos:
  \begin{align*}
    (x_0, y_0)&=(-1,0), &\quad (x_2, y_2)&=(1,2),\\ 
    (x_1, y_1)&=(0,2),   &\quad (x_3, y_3)&=(2,6).
  \end{align*}
  La base de Lagrange asociada al soporte $S=\{x_0,x_1,x_2,x_3\}=\{-1,0,1,2\}$ es:
  \begin{small}
    \begin{align*}
      L_0(x) & =
      \frac{(x-x_1)(x-x_2)(x-x_3)}{(x_0-x_1)(x_0-x_2)(x_0-x_3)} =
      \frac{(x-0)(x-1)(x-2)}{(-1-0)(-1-1)(-1-2)} =
      -{{\left(x-2\right)\,\left(x-1\right)\,x}\over{6}}, \\
      \noalign{\medskip} L_1(x) & =
      \frac{(x-x_0)(x-x_2)(x-x_3)}{(x_1-x_0)(x_1-x_2)(x_1-x_3)} =
      \frac{(x+1)(x-1)(x-2)}{(0+1)(0-1)(0-2)} =
      {{\left(x-2\right)\,\left(x-1\right)\,\left(x+1\right)}\over{2}},
      \\ \noalign{\medskip} L_2(x) & =
      \frac{(x-x_0)(x-x_1)(x-x_3)}{(x_2-x_0)(x_2-x_1)(x_2-x_3)} =
      \frac{(x+1)(x-0)(x-2)}{(1+1)(1-0)(1-2)} =
      -{{\left(x-2\right)\,x\,\left(x+1\right)}\over{2}}, \\
      \noalign{\medskip} L_3(x) & =
      \frac{(x-x_0)(x-x_1)(x-x_2)}{(x_3-x_0)(x_3-x_1)(x_3-x_2)} =
      \frac{(x+1)(x-0)(x-1)}{(2+1)(2-0)(2-1)} =
      {{\left(x-1\right)\,x\,\left(x+1\right)}\over{6}}.
    \end{align*}
  \end{small} 
  Así tenemos el polinomio de interpolación para los valores
  $\{y_0,y_1,y_2,y_3\}=\{0,2,2,6\}$:
  \begin{align*}
    p(x)&= 0\cdot L_0(x) + 2\cdot L_1(x)+ 2\cdot L_2(x)+6\cdot L_3(x)
    \\ &=(x-1)x(x+1)-(x-2)x(x+1)+(x-2)(x-1)(x+1)
    =x^3-x^2+2.
  \end{align*}
  Podemos comprobar que no hubo errores, pues efectivamente $p(x)$ es
  un polinomio que interpola los valores
  $(x_i,y_i)$ (el único de $\Pol_3[x]$ que lo hace):
  \begin{align*}
    p(x_0)&=p(-1)=(-1)^3-(-1)^2+2=0=y_0, & p(x_2)&=1^3-1^2+2 = 2=y_2, \\
    p(x_1)&=p(0) = 2=y_1, & p(x_3)&=2^3-2^2+2 = 6=y_3.
  \end{align*}
\end{example}
% \begin{equation}
%     \{ (x_0,y_0),
%     (x_1,y_1),
%     (x_2,y_2),
%     (x_3,y_3),
%     (x_4,y_4) \} =
%     \{ (2,3),(4,-1),(6,4),(8,0),(10,-1) \}
%   \end{equation}
%   es 
%   $$\frac{5x^4}{64} 
%   - \frac{31x^3}{16}
%   + \frac{265x^2}
  
%   $$
% \
%end{example}
\subsubsection{Fórmula de las diferencias divididas de Newton}
\label{sec:formula-de-newton}


%%% Local Variables:
%%% mode: latex
%%% TeX-master: "../apuntes-MNII.tex"
%%% End: 
