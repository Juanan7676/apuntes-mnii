\chapter[Integración numérica]{Integración numérica%
\footnote{\licenseInfo}}
\label{cha:integracion-numerica}

El cálculo de una integral definida es un problema que aparece con
frecuencia, tanto en matemáticas como en otro ámbitos (ciencia,
ingeniería, economía,...). En concreto, dada una función
$f:[a,b]\subset\Rset\to\Rset$, se trata de aproximar numéricamente su
integral definida en $[a,b]$,
\begin{equation*}
  \int_a^bf(x)\,dx.
\end{equation*}

La necesidad de la aproximación numérica de integrales definidas
estriba en que, con frecuencia, resulta imposible (o demasiado costoso
computacionalmente) su cálculo de forma exacta. No sólo porque, en
contextos experimentales, la función podría venir dada por una tabla
de datos, sino porque la mayor parte de las funciones matemáticas no
son integrables mediante las técnicas usuales del cálculo elemental
y por tanto no resulta aplicable la regla de Barrow, $\int_a^b f(x)\,
dx=F(b)-F(a)$.

En este tema se plantearán métodos o fórmulas numéricas, llamadas
fórmulas de cuadratura, para la aproximación de $\int_a^b f(x)\,
dx$. La primera idea es la sustitución de $f(x)$ por un interpolador,
$p_n(x)$, de forma que $\int_a^bf(x)\,dx \approx \int_a^b p_n(x)\,
dx$. Esta última integral es fácil de calcular y, como veremos,
expresar como $\sum_{i=0}^n \omega_i \, f(x_i)$ (donde los coeficiente
$\omega_i$ provienen, por ejemplo, de la integración de las
funciones base de Lagrange).


Por tanto, las fórmulas de cuadratura para la aproximación de
$\int_a^b f(x)\, dx$ serán expresiones que utilizan los valores de la
función en un conjunto de $n+1$ \emph{nodos}, $\{x_0,x_1,\dots,x_n\}$,
multiplicados por $n+1$ \emph{pesos}
$\{\omega_0,\omega_1,\dots,\omega_n\}$:
\begin{definition}[Fórmula de cuadratura]
  \label{def:formula-cuadratura}
  Una \resaltar{fórmula de cuadratura} (abreviadamente, f.c.) con
  $n+1$ nodos $\{x_i\}_{i=0}^n$ y pesos $\{\omega_i\}_{i=0}^n$ es una
  expresión del tipo:
  \begin{align}
    \label{eq:f.cuadratura}
    I_n(f)&=\sum_{i=0}^n \omega_i \, f(x_i).
    \\
    \intertext{Al valor}
    \notag
    E_n(f) &= \int_a^b f(x) - I_n(f) 
  \end{align}
  se le llama \resaltar{error} de la fórmula de cuadratura. 
  % donde $I_n(f)\in\Rset$ representa un valor aproximado de $\int_a^b
  % f(x)\,dx$ y $E_n(f)\in\Rset$ es el error que se comete en la
  % aproximación.
\end{definition}
En el sentido de que intentaremos que el error <<sea pequeño>>, escribimos:
\begin{equation*}
  I_n(f) \approx \int_a^b f(x)\, dx.
\end{equation*}

\begin{definition}[Orden de una fórmula de cuadratura]
  \label{def:2}
  Una fórmula de cuadratura se dirá de \resaltar{orden} $m\in\Nset$ si
  verifica:
  \begin{align*}
    E_n(p)&=0 \quad \forall p\in\Pol_m[x] \text{ (polinomios
      de grado menor o igual que $m$) y}
    \\
    E_n(p)&\neq 0 \quad \text{para \emph{algún} polinomio }
    p\in\Pol_{m+1}[x] \text{ (de grado $m+1$)}.
  \end{align*}
\end{definition}
Así, decimos que una fórmula de cuadratura de orden $m$ es exacta para
los polinomios de orden (menor o igual a) $m$, pero no para los de
orden $m+1$.
\begin{remark}
  \label{rk:5}
  Utilizando la linealidad con respecto a $f$ de $\int_a^bf(x)\,dx$ y
  de $I_n(f)$, es evidente que una f.c. es de orden $m\in\Nset$ si
  y sólo si verifica:
  \begin{align*}
    E_n(x^k)&=0 \quad \forall k=0\dots,m \text{ y}
    \\
    E_n(x^{m+1})&\neq 0.
  \end{align*}
  Esta suele ser la forma de estudiar, en la práctica, el orden de una
  fórmula de cuadratura.
\end{remark}

\begin{example}
  \label{ex:formula-punto-medio}
  La siguiente fórmula de cuadratura de un punto ($n=0$) es conocida
  como \resaltar{fórmula del punto medio},
  \begin{equation}
    \label{eq:f.c.-pto-medio}
    I_0(f)= (b-a) f\left(\frac{a+b}{2}\right),
  \end{equation}
  y tiene orden $1$. En efecto, evaluando $E_0(x^k)=\int_a^b
  x^k\, dx-I_0(x^k)$ para $k\ge 0$:
  \begin{align*}
   \text{Para } k&=0:\ E_0(1) = \int_a^b 1\cdot dx - (b-a)\cdot 1 = 0.
   \\
   \text{Para } k&=1:\ E_0(x) = \int_a^b x\cdot dx - (b-a)\frac{a+b}{2}
   = \frac{1}{2}(b^2-a^2) -  (b-a) \frac{a+b}{2} = 0.
   \\
   \text{Para } k&=2:\ E_0(x) = \int_a^b x^2\cdot dx 
   - (b-a) \frac{(a+b)^2}{4} 
   = \frac{1}{3}(b^3-a^3) - (b-a)\frac{(a+b)^2}{4} 
   \neq 0.
  \end{align*}
\end{example}

\begin{example}
  \label{ex:formula-trapecio}
  La siguiente f.c. de dos puntos ($n=1$) es conocida
  como \resaltar{fórmula del trapecio}:
  \begin{equation*}
    I_1(f)= \frac{b-a}{2}\left(f(a)+f(b)\right).
  \end{equation*}
  La formula del trapecio es de orden 1:
  \begin{align*}
   \text{Para } k&=0:\ E_1(1) =  (b-a) - \frac{b-a}2\cdot 2 = 0.
   \\
   \text{Para } k&=1:\ E_1(x) = \frac{b^2-a^2}{2} -
   \frac{(b-a)}{2}(a+b) = 0.
   \\
   \text{Para } k&=2:\ E_1(x) = \frac{b^3-a^3}{3} - \frac{(b-a)}{2}(a^2+b^2) 
   \neq 0.
  \end{align*}
\end{example}

\begin{example}
  \label{ex:formula-simpson}
  La \resaltar{fórmula de Simpson} es la siguiente f.c.
  de tres puntos ($n=2$):
  \begin{equation}
    I_2(f)= \frac{b-a}{6}\left[f(a)+f
      \left(\frac{a+b}{2}\right)+f(b)\right],
    \label{eq:formula-simpson}
  \end{equation}
  Su orden es $3$. La demostración es similar a los ejemplos anteriores.
\end{example}

\section{Fórmulas de cuadratura de tipo interpolatorio}
\label{sec:cuadratura-interpolatorio}

La expresión~(\ref{eq:f.cuadratura}) define a una fórmula de
cuadratura para pesos $\omega_i$ y nodos $x_i$ cualesquiera, no
necesariamente provenientes de la interpolación.
Cuando una fórmula de cuadratura viene determinada a partir de la integral del
polinomio de interpolación en un conjunto de nodos, entonces diremos
que se trata de una fórmula de tipo interpolatorio. Específicamente:
\begin{definition}[Fórmula de cuadratura de tipo interpolatorio]
  Sea $f\in C^0([a,b])$ y sea $p_n$ el polinomio de interpolación de
  Lagrange asociado a $f$ y a $n+1$ nodos
  $S=\{x_0,x_1,\dots,x_n\}$. Diremos que una f.c. es de tipo
  interpolatorio si 
  \begin{equation}
    \label{eq:fcti}
    I_n(f)=\int_a^b p_n(x)\, dx.
  \end{equation}
  \label{def:fcti}
\end{definition}

La expresión~(\ref{eq:fcti}) se puede escribir de la forma
de~(\ref{eq:f.cuadratura}), es decir como suma de $f(x_i)$ por
cierto pesos. En efecto, dada $f\in C^0([a,b])$, podemos escribir
$p_n$ usando la fórmula de interpolación de Lagrange:
\begin{equation*}
  p_n(x)=\sum_{i=0}^n f(x_i) L_i(x),
\end{equation*}
donde $\{L_i\}_{i=0}^n$ es la base de Lagrange asociada a
$\{x_i\}_{i=0}^n$. Por lo tanto,
\begin{equation*}
  \int_a^bp_n(x)\,dx = \sum_{i=0}^n f(x_i) \int_a^b L_i(x)\,dx 
  =\sum_{i=0}^n \omega_i f(x_i), \quad \text{con } \omega_i=\int_a^b L_i(x)\,dx.
\end{equation*}

\begin{example}
  \label{ex:formula-pto-medio-interpol}
  La f.c. del punto medio~(\ref{eq:f.c.-pto-medio}) es de tipo
  interpolatorio. En efecto, si $p_0(x)$ es el polinomio de 
  interpolación de grado cero asociado a $f$ en $x_0=(a+b)/2$, es
  decir $p_0(x)=f(x_0)$ para todo $x\in[a,b]$, entonces:
  \begin{equation*}
    I_0(f) = (b-a)f\left(\frac{a+b}{2}\right) = \int_a^b p_0(x)\,dx.
  \end{equation*}
\end{example}

\begin{example}
  \label{ex:formula-trapecio-interpol}
  Fijamos $n=1$, $x_0=a$ y $x_1=b$. Buscamos coeficientes, $\omega_0$
  y $\omega_1$ tales que la siguiente fórmula de cuadratura sea de
  tipo interpolatorio:
  \begin{equation*}
    I_1(f)=\omega_0 f(x_0) + \omega_1 f(x_1).
  \end{equation*}
  Según la definición~\ref{def:fcti}, esto significa que
  \begin{equation*}
    I_1(f)=\int_a^b p_n(x)= 
    \underbrace{\left(\int_a^bL_0(x)\right)}_{\omega_0} f(x_0)+
    \underbrace{\left(\int_a^bL_1(x)\right)}_{\omega_1} f(x_1).
  \end{equation*}
  Como
  \begin{equation*}
    L_0(x)=\frac{x-x_1}{x_0-x_1}=\frac{x-b}{a-b} \quad \text{y} \quad
    L_1(x)=\frac{x-x_0}{x_1-x_0}=\frac{x-a}{b-a},
  \end{equation*}
  entonces
  \begin{equation*}
    \omega_0 = \int_a^b \frac{x-b}{a-b} = \frac{-(a-b)^2}{2(a-b)}=\frac{b-a}{2},
    \quad
    \omega_1 = \int_a^b \frac{x-a}{b-a} = \frac{(b-a)^2}{2(b-a)}=\frac{b-a}{2}.
  \end{equation*}
  Es decir, que la fórmula de cuadratura que buscamos es la
  fórmula del trapecio:
  \begin{equation*}
    I_1(f)=\frac{b-a}{2}
    \left(
      f(a)+f(b)
    \right).
  \end{equation*}
\end{example}

\begin{example}
  La fórmula de Simpson~(\ref{eq:formula-simpson}) es de tipo
  interpolatorio. La demostración se deja como ejercicio.
\end{example}


%%% Local Variables:
%%% mode: latex
%%% TeX-master: "../apuntes-MNII.tex"
%%% End: 
