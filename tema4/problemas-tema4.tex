\documentclass[11pt]{article}
\usepackage[spanish]{babel}
\usepackage[utf8]{inputenc}

% ------ Cargar estilo específico para la relación de problemas
\usepackage{problemas-MNII}

% ==============
\begin{document}
% =============
\begin{flushright}
  \LARGE\it Relación de problemas. Tema \huge 3.\\
  \bigskip
\end{flushright}

\begin{problemas}

  \begin{problema}
    Deducir la fórmula compuesta de los trapecios, así como su
    expresión del error. Utilizando esta fórmula, calcular una
    aproximación de la integral
    $$
    \int_0^1 \log(1+x^2)\, dx
    $$
    en la que podamos garantizar que el error de cuadratura es menor que
    $\varepsilon=0.001$. ¿En cuántos subintervalos hay que dividir el
    intervalo $[0,1]$?
  \item Se considera la fórmula de cuadratura
    \begin{equation*}
      \int_0^1 f(x)\,dx \approx c_0 f(0) + c_1 f\bigg(\frac{1}{3}\bigg)
      + c_2 f\bigg(\frac{2}{3}\bigg) + c_3 f(1).
    \end{equation*}
    \begin{itemize}
    \item Calcular los coeficientes $c_0$, $c_1$, $c_2$ y $c_3$ para
      los que la fórmula de cuadratura es exacta para polinomios de
      grado menor o igual que $3$. Comprobar que la fórmula resultante
      no proporcionan el valor exacto para polinomios de orden $4$.
    \item Aplicar esta fórmula de cuadratura para aproximar la
      siguiente integral:
      \begin{equation*}
        \int_0^1 \big(-\log(\cos x) \big)\,dx
      \end{equation*}
    \item Realizar el cambio de variables $t=(x-a)/(b-a)$ para
      transformar la integral
      \begin{equation*}
        \int_a^b f(x)\, dx
      \end{equation*}
      en una integral sobre el intervalo $[0,1]$. Utilizar la fórmula
      de cuadratura del apartado anterior para deducir una 
      regla de cuadratura en cualquier intervalo $[a,b]$.
    \item Sabiendo que la regla anterior es la fórmula de cuadratura
      de Newton-Côtes cerrada con $4$ nodos, obtener una expresión que
      de la fórmula compuesta correspondiente, que permita aplicarla.
    \item Aplicar la fórmula compuesta con $20$ subintervalos para
      aproximar la siguiente integral:
      \begin{equation*}
        \int_0^{10} \log\big(1+3e^{-x^2}\big)\,dx.
      \end{equation*}
    \end{itemize}
  \end{problema}
  
  \begin{problema}
  Determinar las constantes $A_0$, $A_1$ y $A_2$ de modo que la
  fórmula de cuadratura
  \begin{equation*}
    \int_0^h f(x)\,dx = h \bigg\{
    A_0f(0) + A_1f\bigg(\frac{h}{3}\bigg) + A_2 f(h) \bigg\}
  \end{equation*}
  sea exacta para polinomios del mayor grado posible. Utilizarla para
  aproximar la integral
  \begin{equation*}
    \int_0^{1/2} e^{-x^2}\, dx.
  \end{equation*}
  \end{problema}
 
  \begin{problema}
    Determinar los coeficientes $\alpha_0$, $\alpha_1$,  $\beta_0$ y
    $\beta_1$ para los que la fórmula de cuadratura
    \begin{equation*}
      \int_a^b f(x)\, dx \approx \alpha_0 f(a) + \alpha_1 f(b)
      + \beta_0 f'(a) + \beta_1 f'(b)
    \end{equation*}
    tiene el mayor orden posible. ¿Cuál es este orden de precisión?.
    Sabiendo que el error puede escribirse como
    \begin{equation*}
      \gamma (b-a)^5 f^{4)}(\xi),
    \end{equation*}
    para algún $\xi\in(a,b)$, determinar el valor de $\gamma$.
  \end{problema}

  \begin{problema}
    Para calcular un valor aproximado de $\ln 2$, se utilizará el
    hecho de que la integral
    \begin{equation*}
      \int_1^2 \frac{1}{x}\, dx
    \end{equation*}
    concide con este valor. Averiguar el número de subintervalos
    necesarios para que, al usar la fórmula compuesta de Simpson, el
    error sea inferior a $10^{-3}$. Utilizando el ordenador, calcular
    una aproximación de $\log 2$ y comprobar el error.
  \end{problema}

  \begin{problema}
    Se considera la fórmula de cuadratura
    \begin{equation*}
      \int_0^1 f(x)\, dx \approx A (f(x_0) + f(x_1)).
    \end{equation*}
    \begin{enumerate}
    \item Hallar el valor de $A$, $x_0$ y $x_1$ para que la
      fórmula sea exacta para polinomios del mayor grado
      posible. ¿Cuál es éste?
    \item Comprobar, mediante el cambio de variables $t=2x-1$, que
      la fórmula de cuadratura coincide con la regla de Gauss con
      dos nodos:
      \begin{equation*}
        \int_{-1}^1 g(x)\,dx \approx g\bigg(\frac{-1}{\sqrt
          3}\bigg) + g\bigg(\frac{1}{\sqrt 3}\bigg).
      \end{equation*}
    \end{enumerate}
  \end{problema}
  
  \begin{problema}
    La regla de Lobatto es una fórmula de cuadratura gaussiana
    definida en $[-1,1]$ de la siguiente forma:
    \begin{equation*}
      \int_{-1}^1 f(x)\, dx = A_0 f(-1) + \sum_{i=1}^{n-1} A_if(x_i) +
      A_nf(1).
    \end{equation*}
    \begin{itemize}
    \item Detallar la expresión de la regla de Lobatto para
      $n=3$. ¿Cuál es el orden de precisión de esta regla?
    \item Realizando un cambio de variable adecuado, utilizar la regla
      de Lobatto (con $n=3$) para estimar
      \begin{equation*}
        \int_1^2 e^x\, dx.
      \end{equation*}
    \end{itemize}
  \end{problema}
\end{problemas}

\end{document}

% ===============

%%% Local Variables: 
%%% mode: latex
%%% TeX-master: t
%%% End: 
