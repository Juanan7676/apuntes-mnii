\documentclass[11pt]{article}
\usepackage[spanish]{babel}
\usepackage[utf8]{inputenc}

% ------ Cargar estilo específico para la relación de problemas
\usepackage{problemas-MNII}

% ==============
\begin{document}
% =============
\begin{flushright}
  \LARGE\it Relación de problemas. Tema \huge 4.\\
  \bigskip
\end{flushright}

\begin{problemas}

  \begin{problema}
    Consideremos el siguiente problema de valor inicial:
    \begin{align*}
      &y'=\frac{2y}{x} + x^2e^x, \quad x\in[1,2],\\
      &y(1)=0.
    \end{align*}
    Aproximar la solución con $h=0.1$ usando:
    \begin{enumerate}
    \item El método de Euler.
    \item El método de Euler--Cauchy (o Euler modificado).
    \end{enumerate}
    Comparar, en ambos casos, con la solución aproximada que
    proporciona Python (función \texttt{odeint} en
    \texttt{scipy.integrate}).
  \end{problema}
  
  \begin{problema}%% problema 3
    Consideremos el siguiente problema de valor inicial:
    \begin{align*}
      &y'=1-y^2,\\
      &y(0)=0,
    \end{align*}
    cuya solución exacta es:
    $$ y(t) = \frac{e^{2t}-1}{e^{2t}+1}.$$
    Aproximar la solución con $h=0.1$ usando:
    \begin{enumerate}
    \item El método de Euler.
    \item El método de Euler-Cauchy (o Euler
      modificado).
    \item El método de Runge-Kutta de orden $4$
    \end{enumerate}
  \end{problema}

  \begin{problema}
    Aproximar la solución del problema de Cauchy
    \begin{equation*}
      \left\{
      \begin{aligned}
        &y'=x^2 - \sqrt{y-e^{x/2}}, \quad x\in[0,2], \\
        &y(0)=5
      \end{aligned}
      \right.
    \end{equation*}
    Mediante:
    \begin{enumerate}
    \item El método de Adams--Bashforth de orden $2$.
    \item El método predictor-corrector $AB2$--$AM2$
    \end{enumerate}
  \end{problema}
  
  \begin{problema}
    \begin{enumerate}
    \item Deducir la expresión del método de Adams--Moulton de dos
      pasos:
      \begin{equation*}
        y_{n+2}-y_{n+1} = \frac{h}{12} \big[
        5f(t_{n+2},y_{n+2}) + 8 f(t_{n+1},y_{n+1}) - f(t_{n},y_{n}) \big].
      \end{equation*}

  \item Indicar cuál es, en la etapa $n$--ésima, la expresión del
    método predictor-corrector que se construye mediante el esquema
    anterior y el método de Adams--Bashforth de dos pasos: 
    \begin{equation*}
      y_{n+2}-y_{n+1} = \frac{h}{2} \big[
      3f(t_{n+1},y_{n+1}) - f(t_{n},y_{n}) \big],
    \end{equation*} 
    \end{enumerate}
  \item Corregir y completar el siguiente extracto de un programa Python:
\begin{verbatim}
    # (2) Iteraciones del método predictor-corrector AB2-AM2
    y0 = sol[0]; y1 = sol[0];
    f0 = f(a, y0); f1 = f(a+h, y1);
    for i in (...):
        yn_predic = yn + h/2 * (...)
        yn_correc = (...)
        sol.append(yn)
        tn = soporte[i+1]
        f0 = f1; f1 = f(tn, yn)
    return soporte, sol
\end{verbatim}
  \end{problema}
  \end{problemas}
\end{document}

%%% Local Variables: 
%%% mode: latex
%%% TeX-master: t
%%% End: 
